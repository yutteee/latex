\documentclass{ltjsarticle}
\usepackage{amsmath}
\usepackage{amssymb}
\usepackage{graphicx} % Required for inserting images
\usepackage{enumerate}
\usepackage[version=4]{mhchem}
\usepackage{caption}
\usepackage{url}

\title{実験レポテンプレ}
\author{中村 優作}
\date{April 2023}

\pagestyle{plain}
\begin{document}
\flushleft{
    \huge{令和5年度 化学生命工学実験3}
    
    \vspace{100pt}
    
    \huge{グリニャール試薬を用いたクロスカップリング反応}
}

\vfill
\begin{flushright}

        \Large{\underline{学籍番号 : 082110424}}

        \vspace{30pt}
    
        \Large{\underline{氏名 : 中村優作}}

        \vspace{30pt}
    
        \Large{\underline{実施日 : 10/26, 27, 31 }}
        
\end{flushright}

\newpage

\section{目的}

\quad 遷移金属触媒を用いたクロスカップリング反応は、有機合成において有用なツールである。本実験では、\ce{Ni}触媒とグリニャール試薬を用いた熊田-玉尾カップリングにより、2,5-ジフェニルチオフェンを合成し、得られた化合物について分析することを目的とする。

\section{操作}

\subsection{準備}

\quad $100\mathrm{mL}$2口丸底フラスコに、マグネティックスターラーとマグネシウム$0.16\mathrm{g}$($6.6\mathrm{m mol}$)を投入した。丸底フラスコの上にジムロート冷却管を連結し、ジムロート冷却管の上に\ce{N2}を満たしたバルーンを取り付けた。また、もう一方の口にはセプタムラバーを取り付けた。ダイアフラムポンプにによりフラスコ内の空気を送り出し、その後コックを回転させ、バルーン内の\ce{N2}を導入した。この操作を3回繰り返し、フラスコ内に\ce{N2}を満たした。

\subsection{グリニャール試薬の精製}

\quad $12\mathrm{mL}$シリンジで、$1.32\mathrm{M}$のブロモベンゼンのTHF溶液$5.0\mathrm{mL}$を吸引した。操作2.1で用意したフラスコに、ゴム製セプタムを通してTHF溶液を$1\mathrm{mL}$滴下した。反応が始まるまで激しく攪拌し、残りのTHF溶液をすべて加えた。その後マグネシウムが溶け切るまで30分ほど攪拌を続けたところ、黄褐色の透明な溶液が調整された。ここで、攪拌を激しくしたことにより、少量のマグネシウムは壁面に残っていた。

\subsection{ニッケル触媒におけるクロスカップリング反応}

\quad $12\mathrm{mL}$シリンジで、操作2.1.2の反応溶液を吸引した。$0.75\mathrm{M}$の2,5-ジオブロモチオフェンのTHF溶液を$4.0\mathrm{mL}$を別の$2\mathrm{mL}$シリンジで加えた。窒素ガスを少し流出した。\ce{NiCl2}を$17.1\mathrm{mg}$をセプタムを外して加え、蓋をし、窒素ガスのコックを元の位置に戻した。ここでグリニャール試薬を滴下して、30分間$90\mathrm{^{\circ}C}$程度で加熱した。その後一日攪拌をしたところ、濃い茶色の液体が得られた。$10\mathrm{mL}$\ce{NH4Cl}でクエンチした。酢酸エチル$20\mathrm{mL}$を加え、吸引濾過をしたところ、不純物は発生していなかった。分液ろうとに溶液を移して有機層を回収し、水層を酢酸エチル$10\mathrm{mL}$で洗い、有機層に加えた。この操作を3回繰り返した。得られた有機層を\ce{Na2SO4}で乾燥した。TLCで反応の進行を確認した。その後、\ce{SiO2}を$4.00\mathrm{g}$を加え、1分間攪拌した。濾過をして\ce{Na2SO4}と\ce{SiO2}を除去し、濾液を$100\mathrm{mL}$丸底フラスコに入れてエバポレーションをして、溶媒を除いた。$30\mathrm{mL}$のメタノールを加え、吸引濾過をしたところ、$0.20\mathrm{g}$の固体が得られた。溶液と固体それぞれにUVを照射して観察したところ、固体は発光したが、溶液は発光しなかった。

\section{結果}

\subsection{TLCの確認}

\quad TLCで反応の進行を確認したところ、図\ref{fig:tlc_254}, \ref{fig:tlc_365}のようになった。左に原料、右に反応物、真ん中に両方をプロットした。

\begin{figure}
    \centering
    \begin{minipage}{0.45\textwidth}
        \centering
        \includegraphics[width=2cm]{./data/tlc_254.png}
        \caption{クロスカップリング反応のTLC($254\mathrm{nm}$)}
        \label{fig:tlc_254}
    \end{minipage}
    \begin{minipage}{0.45\textwidth}
        \centering
        \includegraphics[width=2cm]{./data/tlc_365.png}
        \caption{クロスカップリング反応のTLC($365\mathrm{nm}$)}
        \label{fig:tlc_365}
    \end{minipage}
\end{figure}

\subsection{UV照射}

\quad 最終的に得られた固体と溶液について、UV照射をしたところ、図\ref{fig:solid}, \ref{fig:liquid}のようになった。固体は緑に発光したが、溶液は発光しなかった。

\begin{figure}
    \centering
    \begin{minipage}{0.45\textwidth}
        \centering
        \includegraphics[width=4cm]{./data/solid.png}
        \caption{生成物のUV照射}
        \label{fig:solid}
    \end{minipage}
    \begin{minipage}{0.45\textwidth}
        \centering
        \includegraphics[width=4cm]{./data/liquid.png}
        \caption{溶液のUV照射}
        \label{fig:liquid}
    \end{minipage}
\end{figure}

\subsection{得られた個体のNMR}

\quad 図\ref{fig:nmr}に、得られた個体のNMR、帰属を示す。

\begin{figure}
    \centering
    \includegraphics[width=0.8\textwidth]{./data/nmr.png}
    \caption{得られた個体のNMR}
    \label{fig:nmr}
\end{figure}

\quad $\delta7.7(m, 4H, H_c)$, $\delta7.4(s, 2H, H_a)$, $\delta7.4(m, 4H, H_b)$, $\delta7.3(m, 2H, H_a)$のように帰属した。なお、$2.8\mathrm{ppm}$に存在するピークは、洗浄に用いたアセトンによるもの、$2.7$,$2.0\mathrm{ppm}$に観測されたピークは溶媒などに由来する別の物質のピークと考えられる。

\section{設問/課題}

\subsection{この実験のクロスカップリング反応の反応機構を示せ。また、遷移金属触媒を用いたクロスカップリング反応の例を2つ挙げよ。}

\quad 図\ref{fig:coupling}に、本実験でのクロスカップリングの反応機構を示す。クロスカップリングの反応機構は3つの素反応により成り立つ。まず、溶媒中で0価の\ce{Ni}が2,5-ジブロモチオフェンと反応し、Ph-Br結合が切れて\ce{Ni}に結合する(酸化的付加)。次に、\ce{Ni}-\ce{Br}結合とグリニャール試薬が反応し、ハロゲンは反応から出ていく(トランスメタル化)。最後に、\ce{Ni}に付加している2つが\ce{Ni}から脱離しながら結合することにより、生成物が得られる(還元的脱離)。\cite{1}

\begin{figure}
    \centering
    \includegraphics[width=0.8\textwidth]{./data/coupling.png}
    \caption{クロスカップリング反応の反応機構}
    \label{fig:coupling}
\end{figure}

\quad クロスカップリング反応には多くの反応があり、違いは使用するアニール金属試薬である。アリール金属試薬の反応性が高ければ、反応は進行しやすいが使用できる気質が限定されてしまう。そのため、金属試薬の反応性を下げて、利用可能な気質の範囲を広げることを目的にクロスカップリングが研究されてきた。本実験で使用した、熊田-玉尾カップリングは、Grignard試薬を使用しており、反応性が高すぎることが問題点として挙げられている。Grignard試薬は水と激しく反応し少量の水でも発火するため、試薬の長期保存が難しく、反応条件も水や空気をフラスコから完全に除くことが必要となる。\cite{1}

\quad 根岸カップリングは、\ce{Ni}や\ce{Pd}触媒とアリール亜鉛試薬との組み合わせでのカップリング反応である。アリール亜鉛試薬はGrignard試薬に比べて、高い反応性を維持したまま、基質適用範囲を改善でき、副反応も少なく、毒性もほとんどないことが利点となる。アリール亜鉛試薬はアリールGrignardやLi試薬と\ce{ZnCl2}を混ぜて調整される。しかし、この反応も水や空気をフラスコから除くことが要求される。また、カルボン酸、ヒドロキシル基、アルデヒド基など酸性度の高い水素や反応性の高い官能機が存在する気質は使用できない。\cite{1}

\quad 鈴木-宮浦カップリングは、ホウ素試薬を使用するカップリングである。ホウ素試薬は水や酸素に非常に安定で長期保存が可能であり、空気中で反応させることができる。ホウ素が安定であることは、反応が進行しにくいとも言えるが、塩基を添加し、ホウ素の安定性を向上させることで克服している。塩基はホウ素の空のp軌道に配位し、Ate錯体と呼ばれる錯体を作り、トランスメタル化の反応性をあげている。\cite{1}

\subsection{過剰量の臭化フェニルマグネシウムと次の3種類の求電子剤(アセトアルデヒド、酢酸エチル、アセトン)との反応によって得られる生成物をそれぞれ示せ。また、これらの求電子剤を反応性の高い順に並べよ。}

\quad 臭化フェニルマグネシウムと3種類の求電子剤の反応機構を図\ref{fig:asetoarudehido}, \ref{fig:aseton}, \ref{fig:sakusanetiru}に示す。\cite{2}\cite{3}

\begin{figure}
    \centering
    \includegraphics[width=8cm]{./data/asetoarudehido.png}
    \caption{臭化フェニルマグネシウムとアセトアルデヒドの反応機構}
    \label{fig:asetoarudehido}
\end{figure}

\begin{figure}
    \centering
    \includegraphics[width=8cm]{./data/aseton.png}
    \caption{臭化フェニルマグネシウムとアセトンの反応機構}
    \label{fig:aseton}
\end{figure}

\begin{figure}
    \centering
    \includegraphics[width=8cm]{./data/sakusanetiru.png}
    \caption{臭化フェニルマグネシウムと酢酸エチルの反応機構}
    \label{fig:sakusanetiru}
\end{figure}

\quad これらの求電子剤を反応性の高い順に並べると、アセトン、酢酸エチル、アセトアルデヒドとなる。アルコールにおいて、級数が高いほど安定性が増すため、第二級アルコールが生成されるアセトアルデヒドが最も反応性が低いと考えられる。また。アセトンと酢酸エチルでは、最終的な生成物での立体障害が大きいと考えられる酢酸エチルの方が反応性が低いと考えられる。

\section{考察}

\quad NMRの帰属から、得られた生成物は2,5-ジフェニルチオフェンであると考えられる。$0.2\mathrm{g}$の固体が得られたが、理論値は$0.71\mathrm{g}$であるため、収率は$28\%$であった。収率が低い原因として、以下のことが考えられる。

\begin{enumerate}
    \item クロスカップリング反応で反応が完全には終了していない。
    
    \quad TLCの結果からもわかるように、生成物のレーンで原材料の点が少し残っている。このことから、完全には反応が終了していないといえる。グリニャール試薬の精製時に少量のマグネシウムが壁面に残っていたことから、グリニャール試薬が十分に精製されなかった可能性がある。

    \item 不純物が生成し、水層に溶けてしまった。

    \quad 2等量でグリニャール試薬が反応する必要があるため、グリニャール試薬が十分でない場合、2,5-ブロモチオフェンの片方にのみクロスカップリング反応が進行する可能性がある。この場合、生成物の極性は大きく、水層に溶けてしまう可能性がある。また、少量のマグネシウムが壁面に残っていたことから、2,5-ブロモチオフェンがグリニャール試薬として生成し、2,5-ブロモチオフェン同士で反応が進行してしまった可能性も考えられる。
\end{enumerate}

\subsection{なぜ\ce{N2}環境で反応させる必要があるのか}

\quad グリニャール試薬は求核性が強く、水や二酸化炭素と反応する。水と激しく反応することで、金属水酸化物とアルカンに加水分解される。\cite{2}また、二酸化炭素と炭酸塩化反応を起こし、カルボン酸塩を生成する。これは、グリニャール試薬が\ce{CO2}の求電子的炭素に付加し、プロトン化によってカルボン酸マグネシウム塩からカルボン酸が遊離することによって起こる。\cite{4}空気中ではこれらの反応が副反応として起こる可能性があるため、\ce{N2}環境で反応させる必要がある。

\begin{thebibliography}{99}
  \bibitem{1} クロスカップリング入門, 東京工業大学, 秋山勝宏, \url{https://www.jstage.jst.go.jp/article/jaesjb/53/7/53_499/_pdf}
  \bibitem{2} ボルハルトショアー現代有機化学第8版上, 8章, 化学同人
  \bibitem{3} ボルハルトショアー現代有機化学第8版下, 20章, 化学同人
  \bibitem{4} ボルハルトショアー現代有機化学第8版下, 19章, 化学同人
\end{thebibliography}

\end{document} 