\documentclass{ltjsarticle}
\usepackage{amsmath}
\usepackage{amssymb}
\usepackage{graphicx} % Required for inserting images
\usepackage{enumerate}
\usepackage[version=4]{mhchem}
\usepackage{caption}
\usepackage{url}

\title{実験レポテンプレ}
\author{中村 優作}
\date{April 2023}

\pagestyle{plain}
\begin{document}
\flushleft{
    \huge{令和5年度 化学生命工学実験3}
    
    \vspace{100pt}
    
    \huge{グリニャール試薬を用いたクロスカップリング反応、ルミノールによる化学発光}
}

\vfill
\begin{flushright}

        \Large{\underline{学籍番号 : 082110424}}

        \vspace{30pt}
    
        \Large{\underline{氏名 : 中村優作}}

        \vspace{30pt}
    
        \Large{\underline{実施日 : 10/26, 27, 31 }}
        
\end{flushright}

\newpage

\section{目的}
\section{操作}

\subsection{グリニャール試薬を用いたクロスカップリング反応}

\subsubsection{準備}

\quad $100\mathrm{mL}$2口丸底フラスコに、マグネティックスターラーとマグネシウム$0.16\mathrm{g}$($6.6\mathrm{m mol}$)を投入した。丸底フラスコの上にジムロート冷却管を連結し、ジムロート冷却管の上に\ce{N2}を満たしたバルーンを取り付けた。また、もう一方の口にはセプタムラバーを取り付けた。ダイアフラムポンプにによりフラスコ内の空気を送り出し、その後コックを回転させ、バルーン内の\ce{N2}を導入した。この操作を3回繰り返し、フラスコ内に\ce{N2}を満たした。

\subsubsection{グリニャール試薬の精製}

\quad $12\mathrm{mL}$シリンジで、$1.32\mathrm{M}$のブロモベンゼンのTHF溶液$5.0\mathrm{mL}$を吸引した。操作2.1で用意したフラスコに、ゴム製セプタムを通してTHF溶液を$1\mathrm{mL}$滴下した。反応が始まるまで激しく攪拌し、残りのTHF溶液をすべて加えた。その後マグネシウムが溶け切るまで30分ほど攪拌を続けたところ、黄褐色の透明な溶液が調整された。ここで、攪拌を激しくしたことにより、少量のマグネシウムは壁面に残っていた。

\subsubsection{ニッケル触媒におけるクロスカップリング反応}

\quad $12\mathrm{mL}$シリンジで、操作2.2の反応溶液を吸引した。$0.75\mathrm{M}$の2,5-ジオブロモチオフェンのTHF溶液を$4.0\mathrm{mL}$を別の$2\mathrm{mL}$シリンジで加えた。窒素ガスを少し流出した。\ce{NiCl2}を$17.1\mathrm{mg}$をセプタムを外して加え、蓋をし、窒素ガスのコックを元の位置に戻した。ここでグリニャール試薬を滴下して、30分間$90\mathrm{^{\circ}C}$程度で加熱した。その後一日攪拌をしたところ、濃い茶色の液体が得られた。$10\mathrm{mL}$\ce{NH4Cl}でクエンチした。酢酸エチル$20\mathrm{mL}$を加え、吸引濾過をしたところ、不純物は発生していなかった。分液ろうとに溶液を移して有機層を回収し、水層を酢酸エチル$10\mathrm{mL}$で洗い、有機層に加えた。この操作を3回繰り返した。得られた有機層を\ce{Na2SO4}で乾燥した。TLCで反応の進行を確認した。その後、\ce{SiO2}を$4.00\mathrm{g}$を加え、1分間攪拌した。濾過をして\ce{Na2SO4}と\ce{SiO2}を除去し、濾液を$100\mathrm{mL}$丸底フラスコに入れてエバポレーションをして、溶媒を除いた。$30\mathrm{mL}$のメタノールを加え、吸引濾過をしたところ、$0.2\mathrm{g}$の固体が得られた。溶液と固体それぞれにUVを照射して観察したところ、固体は発光したが、溶液は発光しなかった。

\subsection{ルミノールによる化学発光}

\subsubsection{3-ニトロフタルヒドラジド(Ⅲ)の合成}

\quad アルミホイルで蓋をしたチューブに、2-ニトロフタル酸無水物$0.5\mathrm{g}$と酢酸$2\mathrm{mL}$を加えた。その後、$120\mathrm{^{\circ}C}$で加熱し、完全に溶解するまで攪拌したところ、黄色の透明な溶液が得られた。室温になるまで冷却し、ヒドラジン水和物$0.14\mathrm{mL}$を混合物が固まらないように激しく振りながら攪拌した。$120\mathrm{^{\circ}C}$で加熱しながら10分攪拌した。その後室温まで冷却し、水で複数回、メタノールで一回洗浄しながら吸引濾過をしたところ、黄淡色の結晶が得られた。結晶の重さを測ったところ、$0.285\mathrm{g}$であった。

\subsubsection{3-アミノフタルヒドラジド(Ⅲ)の合成}

\quad 操作2.2.1で得られた固体をチューブに入れ、質量パーセント濃度が10\%の\ce{NaOH}$2.5\mathrm{mL}$に溶かしたところ、暗赤色の溶液が得られた。白い固体が沈澱しており、スパチュラーで砕いた。その後、亜硫酸水素ナトリウム$1.5\mathrm{g}$を加え、少量の水でチューブの壁面を洗った。攪拌しながら5分間加熱したところ、黄色の溶液が得られた。室温まで冷却した後、$1.0\mathrm{mL}$の酢酸を加えた。流水でチューブを冷却したところ、黄色の沈殿物が形成した。吸引濾過をして、水で洗浄したところ、黄色の固体が得られた。結晶の重さを測ったところ、$0.045\mathrm{g}$であった。

\subsubsection{ルミノールの化学発光}

操作2.2.2で得られた固体を2\%\ce{NaOH}$10\mathrm{mL}$に溶かした。この溶液を水$15\mathrm{mL}$で希釈した。また、別のビーカーに10\%フェリシアン酸カリウム水溶液$2.5\mathrm{mL}$、10\%\ce{H2O2}$2.5\mathrm{mL}$および水$10\mathrm{mL}$を混合したところ、黄色透明の溶液が得られた。これらの溶液を暗所で反応させたところ、青色に発光した。

\section{結果}

\subsection{グリニャール試薬を用いたクロスカップリング反応}

\subsubsection{TLCの確認}

\quad TLCで反応の進行を確認したところ、図\ref{}のようになった。左に原料、右に反応物、真ん中に両方をプロットした。

\subsubsection{UV照射}

\quad 最終的に得られた固体と溶液について、UV照射をしたところ、図\ref{}のようになった。固体は緑に発光したが、溶液は発光しなかった。

\section{考察}
\section{設問/課題}
\section{参考文献}
\begin{thebibliography}{99}
    % \bibitem{1} hogehoge \url{https://www.google.com/}
\end{thebibliography}

\end{document} 