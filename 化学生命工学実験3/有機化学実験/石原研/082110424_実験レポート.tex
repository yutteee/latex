\documentclass{ltjsarticle}
\usepackage{amsmath}
\usepackage{amssymb}
\usepackage{graphicx} % Required for inserting images
\usepackage{enumerate}
\usepackage[version=4]{mhchem}
\usepackage{caption}
\usepackage{url}

\title{実験レポテンプレ}
\author{中村 優作}
\date{April 2023}

\pagestyle{plain}
\begin{document}
\flushleft{
    \huge{化学生命工学実験3 レポート}
    
    \vspace{80pt}
    
    \huge{医薬品リドカインの合成}

    \vspace{25pt}

    \huge{LIDOCAINE AS A SYNTHETIC DRUG}
}

\vfill
\flushleft{
    \Large{\underline{班番号:4班}}
    
    \vspace{15pt}
    
    \Large{\underline{報告者:082110424 中村優作}}

    \vspace{15pt}

    \Large{\underline{共同実験者:戸田明希, 中島志⼈, 中村⾥新}}

    \vspace{15pt}

    \Large{\underline{提出日:2023年11月16日}}

    \vspace{60pt}
}

\newpage

\section{目的}

\quad リドカインは局所麻酔薬として用いられ、皮膚に塗布したり、神経に注射した時の麻酔作用が比較的高く、毒性や副作用の発生率が低いことで注目されている。本実験ではリドカインを合成し、それぞれの合成手順での結果について考察することを目的とする。

\section{実験操作・結果}

\subsection{\ce{SnCl2}還元による2,6-ジメチルアニリンの合成}

\quad $100\mathrm{mL}$三角フラスコに、2,6-ジメチルニトリルベンゼン$2.5\mathrm{g}$($2.25\mathrm{mL}$)を氷酢酸$25\mathrm{mL}$に溶かした。別の$100\mathrm{mL}$三角フラスコに$10\mathrm{g}$\ce{SnCl2}・\ce{2H2O}を$20\mathrm{mL}$塩酸塩に溶かした。調整したこれらの2つの溶液を振り混ぜて混合し、15分間放置した。混合物を冷却し、ブフナー漏斗で結晶塩を集めた。三角フラスコに結晶を移し、$12\mathrm{mL}$の水を加えた。$30\%$\ce{NaOH}を注意深く加えて強塩基性にした。冷却し、ジエチルエーテル$15\mathrm{mL}$と$10\mathrm{mL}$で抽出した。エーテル抽出液を水$10\mathrm{mL}$で2回洗浄した。食塩水$10\mathrm{mL}$で1回洗浄した。\ce{Na2SO4}上で乾燥した。濾過した溶液を$100\mathrm{mL}$丸底フラスコに移した。エバポレーションにより溶液を蒸発させ、黄色の溶液を得た。重量を測定したところ、$0.16\mathrm{g}$であった。

\quad 理論値を計算すると以下のようになった。

\begin{equation}
    \mathrm{Yield} = \frac{2.5\mathrm{g}}{151.17\mathrm{g/mol}} \times 121.18\mathrm{g/mol} = 2.0\mathrm{g}
\end{equation}

したがって、収率は$0.155\mathrm{g}/2.00\mathrm{g} \times 100 = 7.8\%$であった。

\subsection{$\alpha$-クロロ-2,6-ジメチルアセトアニリドの合成}

\quad 試験管に、2,6-ジメチルアニリン$0.35\mathrm{g}$、氷酢酸$1.75\mathrm{mL}$、クロロアセチルクロライド$1.85\mathrm{g}$($1.3\mathrm{mL}$)、オイルバス$40〜50\mathrm{^{\circ}C}$に15〜20分温めた。酢酸ナトリウム$2.5\mathrm{g}$を水$50\mathrm{mL}$溶かした溶液に加えた。冷却し、生成物をブフナー漏斗で集めた。ろうとの中の固体を酢酸臭がなくなるまで水ですすぎ、濾紙に移して風乾させた。重量を測定したところ、$0.375\mathrm{g}$であった。

\quad キシリジンに対して過剰量のクロロアセチルクロライドを用いたため、理論値を計算すると以下のようになった。

\begin{equation}
    \mathrm{Yield} = \frac{0.35\mathrm{g}}{121.18\mathrm{g/mol}} \times 197.67\mathrm{g/mol} = 0.57\mathrm{g}
\end{equation}

したがって、収率$0.375\mathrm{g}/0.571\mathrm{g} \times 100 = 66\%$であった。

\subsection{$\alpha$-ジメチルアミノ-2,6-ジメチルアセトアニリドの合成}

\quad 試験管に、$\alpha$-クロロ-2,6-ジメチルアセトアニリド$0.375\mathrm{g}$、とトルエン$5\mathrm{mL}$を混合した。$0.011\mathrm{g}$のジエチルアミンを加え、オイルバスで$100\mathrm{^{\circ}C}$に温めた。反応の進行を30分ごとに2回TLCで確認した。混合物を冷却して濾過したところ、$0.261\mathrm{g}$の固体が得られた。濾液を分液漏斗に移し、$10\mathrm{mL}$の3M\ce{HCl}で2回抽出した。三角フラスコで水層を冷却した。30\%\ce{NaOH}を$10\mathrm{mL}$加え、溶液を強塩基性にした。$20\mathrm{mL}$のペンタンで抽出し、ペンタン層を$5\mathrm{mL}$の水6回で洗浄した。\ce{Na2SO4}で乾燥した。$50\mathrm{mL}$丸底フラスコに移してエバポレーションを行い濃縮し、生成物の重量を測ったところ、$0.13\mathrm{g}$であった。

\quad TLCの結果、反応液には原料のスポットと生成物のスポットが確認され、完全には反応が進行していなかったと言える。1回目のTLCの$R_f$値を求めると以下のようになった。

\begin{equation}
    R_f(原料) = \frac{2.0\mathrm{cm}}{4.2\mathrm{cm}} = 0.48
\end{equation}

\begin{equation}
    R_f(反応物) = \frac{1.3\mathrm{cm}}{4.2\mathrm{cm}} = 0.31
\end{equation}

\quad また。理論値を計算すると以下のようになった。

\begin{equation}
    \mathrm{Yield} = \frac{0.375\mathrm{g}}{197.67\mathrm{g/mol}} \times 234.35\mathrm{g/mol} = 0.44\mathrm{g}
\end{equation}

したがって、収率は$0.131\mathrm{g}/0.445\mathrm{g} \times 100 = 29\%$であった。

\section{考察及び課題}

\subsection{\ce{SnCl2}還元による2,6-ジメチルアニリンの合成}

\subsubsection{反応機構}

\quad 反応機構を図\ref{fig:3.1.1}に示す。電気陰性度の大きい酸素原子に結合して求電子性が高い状態の窒素原子に対し、スズが求核攻撃をすることで反応が始まる。酸素原子は、溶媒中のプロトンに結合し、水として脱離し、最終的にアミンとなる。

\begin{figure}[htbp]
    \centering
    \includegraphics[width=12cm]{./data/3.1.1.png}
    \caption{\ce{SnCl2}還元による2,6-ジメチルアニリンの合成の反応機構}
    \label{fig:3.1.1}
\end{figure}

\subsubsection{問1の回答}
% SnCl2還元から単離された結晶塩の元素分析結果は以下の通りであった:33.4%C、4.2%H、36.8%Cl、4.9%N、20.6%Sn。その経験式を計算し、構造を提案しなさい
\quad 元素分析の結果から、分子量で割ってモル比を求めたところ、C : H : Cl : N : Sn = 16 : 24 : 6 : 2 : 1となった。反応に用いた2,6-ジメチルアニリンの分子式が\ce{C8H9NO2}であることとから、図\ref{fig:3.1.2}のような構造が考えられる。

\begin{figure}[htbp]
    \centering
    \includegraphics[width=8cm]{./data/3.1.2.png}
    \caption{単離された結晶塩の構造}
    \label{fig:3.1.2}
\end{figure}

\subsubsection{問2の回答}
% SnCl2 還元の平衡方程式を書きなさい。あなたの答えと実際に使用した反応物のモル比を比較しなさい。

\quad \ce{SnCl2}還元の化学反応式は以下のようになる。\ce{Sn}が還元剤として働き、2価から4価に酸化される。

\begin{equation}
    \ce{C8H9NO2 + 2SnCl2 + 6HCl -> C8H11N + 2SnCl4 + 2H2O}
\end{equation}

\quad 実際に使用した反応物のモル比は、2,6-ジメチルニトリルベンゼン : \ce{SnCl2}・\ce{2H2O} = 0.0165 : 0.0443 = 1 : 2.7 であった。理論値は 1 : 2 であるため、過剰量の\ce{SnCl2}・\ce{2H2O}を用いたと考えられる。

\subsubsection{収率が低い原因}

\quad 収率が7.8\%と、極端に低くなってしまった。原因として、溶液を15分間放置して固体を析出させる際、十分に固体が析出しておらず、ブフナー漏斗で回収できなかったことが考えられる。

\subsection{$\alpha$-クロロ-2,6-ジメチルアセトアニリドの合成}

\subsubsection{反応機構}

\quad 反応機構を図\ref{fig:3.2.1}に示す。クロロアセチルクロライドはカルボン酸誘導体であり、アミノ基のような求核剤と置換反応を起こす。\ce{Cl}の電気陰性度は大きく、脱離能が高いため、カルボン酸誘導体の共鳴構造での\ce{Cl}の寄与は小さく、求核剤による反応を受けやすい。\cite{1}

\begin{figure}[htbp]
    \centering
    \includegraphics[width=12cm]{./data/3.2.1.png}
    \caption{$\alpha$-クロロ-2,6-ジメチルアセトアニリドの合成の反応機構}
    \label{fig:3.2.1}
\end{figure}

\subsubsection{問3の回答}
% 合成の第二段階における酢酸ナトリウムの機能は何ですか?反応の完全な式を書きなさい。

\quad 酢酸ナトリウム溶液を加えることで、アミン由来のプロトンが酢酸ナトリウムによって中和されるためだと考えられる。$\alpha$-クロロ-2,6-ジメチルアセトアニリドが共役酸として溶媒中に残るため、図\ref{fig:3.2.2}のように、酢酸イオンが求核剤として働き、プロトンが脱離する。このような中和反応を利用することで目的の化合物を析出させるための役割を果たしていると考えられる。

\begin{figure}[htbp]
    \centering
    \includegraphics[width=10cm]{./data/3.2.2.png}
    \caption{酢酸ナトリウムの働き}
    \label{fig:3.2.2}
\end{figure}

\subsection{$\alpha$-ジメチルアミノ-2,6-ジメチルアセトアニリドの合成}

\subsubsection{反応機構}

\quad 反応機構を図\ref{fig:3.3.1}に示す。ジエチルアミンが求核剤として働き、Sn2反応が起こり、電気陰性度が高く脱離能が高い\ce{Cl-}が脱離する。ケトンのカルボニル炭素の方が求電子性は大きいと考えられるが、ジエチルアミンはかさ高い塩基であり、立体障害的にカルボニル炭素に対して求核攻撃をすることは難しいと考えられる。

\begin{figure}[htbp]
    \centering
    \includegraphics[width=12cm]{./data/3.3.1.png}
    \caption{$\alpha$-ジメチルアミノ-2,6-ジメチルアセトアニリドの合成の反応機構}
    \label{fig:3.3.1}
\end{figure}

\subsubsection{収率が低い原因}

\quad 収率が29\%と、低くなってしまった。原因として、反応が完全に進行していなかったことが考えられる。TLCの結果、反応液には原料のスポットと生成物のスポットが確認され、完全には反応が進行していなかったと言える。そのため、完全に反応が進行したことを確認してから、次の工程に進むことで収率を上げることができたと考えられる。

\subsubsection{溶液を強塩基性にする理由}

\quad この反応はプロトンの脱離を伴う反応である。塩基性条件下では、\ce{OH-}が求核剤として働いてプロトンが脱離する。逆に塩基性条件下でない場合、ジエチルアミンが求核剤として働いてジエチルアミンがプロトン化されたものが生成すると考えられる。この場合、$\alpha$-ジメチルアミノ-2,6-ジメチルアセトアニリドの合成の反応が進行しづらくなるため、溶液を強塩基性にする必要があると考えられる。

\subsubsection{問4の回答}
% 3rdのステップで、還流しているトルエンから結晶化する化合物は何ですか?この反応の平衡方程式を書きなさい。
\quad 3.3.3で述べたように、ジエチルアミンがプロトン化されたものと、塩化物イオンとが結晶を生成したと考えられる。この反応の平衡方程式は以下のようになる。

\begin{equation}
    \ce{C4H12N+ + Cl- <=> C4H12NCl}
\end{equation}

\section{結論}

\quad 目的のリドカインを合成することができた。しかし、全体的に収率が低くなってしまったため、完全に結晶を析出させる、反応が完全に進行したことを確認するなどの工夫が必要であると考えられる。

\begin{thebibliography}{99}
    \bibitem{1} ボルハルト・ショアー現代有機化学第8版下, 20章
\end{thebibliography}

\end{document} 