\documentclass{ltjsarticle}
\usepackage{amsmath}
\usepackage{amssymb}
\usepackage{graphicx} % Required for inserting images
\usepackage{enumerate}
\usepackage[version=4]{mhchem}
\usepackage{caption}
\usepackage{url}

\title{実験レポテンプレ}
\author{中村 優作}
\date{April 2023}

\pagestyle{plain}
\begin{document}
\flushleft{
    \huge{化学生命工学実験3 レポート}
    
    \vspace{80pt}
    
    \huge{医薬品リドカインの合成}

    \vspace{25pt}

    \huge{LIDOCAINE AS A SYNTHETIC DRUG}
}

\vfill
\flushleft{
    \Large{\underline{班番号:4班}}
    
    \vspace{15pt}
    
    \Large{\underline{報告者:082110424 中村優作}}

    \vspace{15pt}

    \Large{\underline{共同実験者:戸田明希, 中島志⼈, 中村⾥新}}

    \vspace{15pt}

    \Large{\underline{提出日:2023年11月16日}}

    \vspace{60pt}
}

\newpage

\section{目的}
\section{実験操作・結果}

\subsection{\ce{SnCl2}還元による2,6-ジメチルアニリンの合成}

\quad $100\mathrm{mL}$三角フラスコに、2,6-ジメチルニトリルベンゼン$2.5\mathrm{g}$($2.25\mathrm{mL}$)を氷酢酸$25\mathrm{mL}$に溶かした。別の$100\mathrm{mL}$三角フラスコに$10\mathrm{g}$\ce{SnCl2}・\ce{2H2O}を$20\mathrm{mL}$塩酸塩に溶かした。調整したこれらの2つの溶液を振り混ぜて混合し、15分間放置した。混合物を冷却し、ブフナー漏斗で結晶塩を集めた。三角フラスコに結晶を移し、$12\mathrm{mL}$の水を加えた。$30\%$\ce{NaOH}を注意深く加えて強塩基性にした。冷却し、ジエチルエーテル$15\mathrm{mL}$と$10\mathrm{mL}$で抽出した。エーテル抽出液を水$10\mathrm{mL}$で2回洗浄した。食塩水$10\mathrm{mL}$で1回洗浄した。\ce{Na2SO4}上で乾燥した。濾過した溶液を$100\mathrm{mL}$丸底フラスコに移した。エバポレーションにより溶液を蒸発させ、黄色の溶液を得た。重量を測定したところ、$0.16\mathrm{g}$であった。

\quad 理論値を計算すると以下のようになった。

\begin{equation}
    \mathrm{Yield} = \frac{2.5\mathrm{g}}{151.75\mathrm{g/mol}} \times 121.18\mathrm{g/mol} = 2.0\mathrm{g}
\end{equation}

したがって、収率は$0.155\mathrm{g}/2.00\mathrm{g} \times 100 = 7.8\%$であった。

\subsection{$\alpha$-クロロ-2,6-ジメチルアセトアニリドの合成}

\quad 試験管に、2,6-ジメチルアニリン$0.35\mathrm{g}$、氷酢酸$1.75\mathrm{mL}$、クロロアセチルクロライド$1.85\mathrm{g}$($1.3\mathrm{mL}$)、オイルバス$40〜50\mathrm{^{\circ}C}$に15〜20分温めた。酢酸ナトリウム$2.5\mathrm{g}$を水$50\mathrm{mL}$溶かした溶液に加えた。冷却し、生成物をブフナー漏斗で集めた。ろうとの中の固体を酢酸臭がなくなるまで水ですすぎ、濾紙に移して風乾させた。重量を測定したところ、$0.375\mathrm{g}$であった。
% TODO:収率

\subsection{$\alpha$-ジメチルアミノ-2,6-ジメチルアセトアニリドの合成}

\quad 試験管に、$\alpha$-クロロ-2,6-ジメチルアセトアニリド$0.375\mathrm{g}$、とトルエン$5\mathrm{mL}$を混合した。$0.011\mathrm{g}$のジエチルアミンを加え、オイルバスで$100\mathrm{^{\circ}C}$に温めた。


\section{考察及び課題}
\section{結論}
\begin{thebibliography}{99}
    % \bibitem{1} hogehoge \url{https://www.google.com/}
\end{thebibliography}
\section{感想等}
\section{リドカインの$^{1}$H NMR}

\end{document} 