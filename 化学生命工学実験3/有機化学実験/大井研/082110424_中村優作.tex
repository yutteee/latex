\documentclass{ltjsarticle}
\usepackage{amsmath}
\usepackage{amssymb}
\usepackage{graphicx} % Required for inserting images
\usepackage{enumerate}
\usepackage[version=4]{mhchem}
\usepackage{caption}
\usepackage{url}

\title{実験レポテンプレ}
\author{中村 優作}
\date{April 2023}

\pagestyle{plain}
\begin{document}
\flushleft{
    \huge{令和5年度 化学生命工学実験3}
    \vspace{100pt}

    \huge{キラル相間移動触媒を利用したフェニルアラニンの不斉合成}
    \vspace{20pt}

    \huge{(±)-シトロネラールの誘導体化}
}

\vfill
\begin{flushright}
        \Large{\underline{学籍番号 : 082110424}}
        \vspace{30pt}

        \Large{\underline{氏名 : 中村優作}}

        \vspace{30pt}
    
        \Large{\underline{実施日 : 2023/11/1, 11/2, 11/6}}
        
\end{flushright}

\newpage

\section{目的}
\section{操作}

\subsection{不斉相間移動触媒反応}

1gと2b

\subsubsection{コアユニットAとサブユニットBの選択}

\subsubsection{キラルPTCsの合成}

\quad 試験管に、ハロゲン化アルキルアルキルB($0.49\mathrm{g}$)と、キナルアルカイドB($0.2\mathrm{m mol}$)を入れた。$2\mathrm{mL}$のアセトンを加え、$50\mathrm{^{\circ}C}$で1時間攪拌した。その後室温まで冷却し、$8\mathrm{mL}$のヘキサンで冷却した。数分間攪拌し、沈殿物を濾過した。得られた沈殿物をヘキサンで洗浄し回収したところ、$0.002\mathrm{g}$の固体が得られた。

\subsubsection{相間移動条件下での不斉ものアルキル化反応}

\quad バイアルに、N-(Diphenylmethylene)glycine tert-butyl esterを($0.29\mathrm{g}$)、1,4-dibromobenzeneを$0.24\mathrm{g}$, 臭化アンモニウム$0.23\mathrm{g}$を加えた。$6.0\mathrm{mL}$のトルエンに溶かした。トルエン中$1.1\mathrm{M}$臭化ベンジル溶液$1.1\mathrm{m mol}$と、50\%水性\ce{KOH}$2.0\mathrm{mL}$を加えたところ、黄色の溶液になった。室温で20時間攪拌したところ、白色の溶液に戻った。$4\mathrm{mL}$の水を加えて反応をクエンチした。ピペットで水層を取り除いき、有機層をシリカゲルカラムに通して生成物を回収した。

\subsubsection{アルキル化生成物の完全脱保護によるフェニルアラニン塩酸酸の調整}

\quad バイアルに、$6\mathrm{M}$\ce{HCl}$4\mathrm{mL}$加え、一日攪拌した。$4\mathrm{mL}$の\ce{H2O}で希釈し、5分間静置した。水層を$100\mathrm{mL}$の丸底フラスコに移し、$50\mathrm{^{\circ}C}$で加熱しながら真空中で濃縮したところ

\subsection{(±)-シトロネラールの誘導体化}

\subsubsection{(±)-シトロネラールのエネ反応}

\quad 

\section{結果}
\section{考察}
\section{設問/課題}
\section{参考文献}
\begin{thebibliography}{99}
    % \bibitem{1} hogehoge \url{https://www.google.com/}
\end{thebibliography}

\end{document} 