\documentclass{ltjsarticle}
\usepackage{amsmath}
\usepackage{amssymb}
\usepackage{graphicx} % Required for inserting images
\usepackage{enumerate}
\usepackage[version=4]{mhchem}
\usepackage{caption}
\usepackage{url}

\title{実験レポテンプレ}
\author{中村 優作}
\date{April 2023}

\pagestyle{plain}
\begin{document}
\flushleft{
    \huge{令和5年度 化学生命工学実験3}
    \vspace{100pt}

    \huge{キラル相間移動触媒を利用したフェニルアラニンの不斉合成}
    \vspace{20pt}

    \huge{(±)-シトロネラールの誘導体化}
}

\vfill
\begin{flushright}
        \Large{\underline{学籍番号 : 082110424}}
        \vspace{30pt}

        \Large{\underline{氏名 : 中村優作}}

        \vspace{30pt}
    
        \Large{\underline{実施日 : 2023/11/1, 11/2, 11/6}}
        
\end{flushright}

\newpage

\section{目的}
\quad 本実験では、不斉相間移動触媒反応と、(±)-シトロネラールの誘導体化を行う。不斉相間移動触媒反応では、キナアルカロイド類似体とハロゲン化アルキルを選択し、生成物のエナンチオマー選択性を算出し、それについて考察することを目的とする。また、シトロネラールの誘導体化では、得られた生成物の収率について考察することを目的とする。

\section{操作}

\subsection{不斉相間移動触媒反応}


\subsubsection{コアユニットAとサブユニットBの選択}

\quad 図\ref{fig:A}に示す構造のキナアルカロイド類似体と、図\ref{fig:B}に示すハロゲン化アルキルを用いて、触媒分子を合成することに決定した。

\begin{figure}[htbp]
    \centering
    \begin{minipage}{0.45\textwidth}
        \centering
        \includegraphics[width=4cm]{./data/operation/core_unit_a.png}
        \caption{選択したキナアルカロイド類似体}
        \label{fig:A}
    \end{minipage}
    \begin{minipage}{0.45\textwidth}
        \centering
        \includegraphics[width=2.5cm]{./data/operation/core_unit_b.png}
        \caption{選択したハロゲン化アルキル}
        \label{fig:B}
    \end{minipage}
\end{figure}

\subsubsection{キラルPTCsの合成}

\quad 試験管に、ハロゲン化アルキルアルキルB($0.49\mathrm{g}$)と、キナルアルカイドA($0.2\mathrm{m mol}$)を入れた。$2\mathrm{mL}$のアセトンを加え、$50\mathrm{^{\circ}C}$で1時間攪拌した。その後室温まで冷却し、$8\mathrm{mL}$のヘキサンで希釈した。数分間攪拌し、沈殿物を濾過した。得られた沈殿物をヘキサンで洗浄し回収したところ、$0.002\mathrm{g}$の固体が得られた。

\subsubsection{相間移動条件下での不斉モノアルキル化反応}

\quad バイアルに、N-(Diphenylmethylene)glycine tert-butyl esterを($0.29\mathrm{g}$)、1,4-dibromobenzeneを$0.24\mathrm{g}$, 臭化アンモニウム$0.23\mathrm{g}$を加えた。$6.0\mathrm{mL}$のトルエンに溶かした。トルエン中$1.1\mathrm{M}$臭化ベンジル溶液$1.1\mathrm{m mol}$と、50\%水性\ce{KOH}$2.0\mathrm{mL}$を加えたところ、黄色の溶液になった。室温で20時間攪拌したところ、白色の溶液に戻った。$4\mathrm{mL}$の水を加えて反応をクエンチした。ピペットで水層を取り除いき、有機層をシリカゲルカラムに通して生成物を回収した。

\subsubsection{アルキル化生成物の完全脱保護によるフェニルアラニン塩酸酸の調整}

\quad バイアルに、$6\mathrm{M}$\ce{HCl}$4\mathrm{mL}$加え、一日攪拌した。$4\mathrm{mL}$の\ce{H2O}で希釈し、5分間静置した。水層を$100\mathrm{mL}$の丸底フラスコに移し、$50\mathrm{^{\circ}C}$で加熱しながら真空中で濃縮した。$10\mathrm{mL}$の酢酸エチルを使用して濾過をし、乾燥して固体を得た。重量を測ったところ、$0.023\mathrm{g}$であった。また、融点を測定したところ$202$〜$210\mathrm{^{\circ}C}$であった。

\subsection{(±)-シトロネラールの誘導体化}

\subsubsection{(±)-シトロネラールのエン反応}

\quad 丸底フラスコに、臭化亜鉛$0.37\mathrm{g}$($1.65\mathrm{m mol}$)とスターラーを入れた。室温でヘキサン$15\mathrm{mL}$を加えた。ドラフト内で、シトロネラール$1.0\mathrm{mL}$($5.5\mathrm{m mol}$)を導入したところ、柑橘系の匂いがした。三方コックと、\ce{N2}で満たしたバルーンをフラスコに取り付けた。室温で一晩攪拌した。その後、調整した溶液の反応の進行をTLCで確認した。ヘキサンで溶液を濾過し、エバポレーションをして固体を得た。重量を測定したところ、$0.50\mathrm{g}$の固体を得た。シリカゲルクロマトグラフィー(固定層としてFuji silysia NH($1\mathrm{g}$)、溶離液として\ce{EtOAc}/ヘキサン=1:10($20\mathrm{mL}$)を使用)をして残渣を精製し、無色の油を得た。得られた生成物をTLCで確認した。

\subsubsection{化合物Cの還元}

\quad 試験管にエタノール$5\mathrm{mL}$を加えた。操作2.2.1で得られた化合物$0.15\mathrm{mL}$と、アゾジカルボン酸カリウム$1.9\mathrm{g}$($10\mathrm{m mol}$)、スターラーを投入した。アルミブロックで$70\mathrm{^{\circ}C}$で加熱し、酢酸$1.2\mathrm{mL}$($20\mathrm{m mol}$)のエタノール$5\mathrm{mL}$を滴下したところ、泡が発生し、黄色の溶液が得られた。一晩溶液を攪拌したところ、溶液は透明となり、白い沈殿が得られた。室温まで冷却し、\ce{NaHCO3}の飽和水溶液$10\mathrm{mL}$で希釈した。酢酸エチルで2回抽出し、brineで洗浄し、\ce{Na2SO4}で乾燥した。濾過をし、溶液をエバポレーションして固体を得た。得られた個体は湿布のような香りがした。重量を測定したところ、$0.20\mathrm{g}$であった。

\section{結果}

\subsection{不斉相間移動触媒反応}

\quad 得られた生成物のPDAクロマトグラムの結果を図\ref{fig:PDA}に示す。

\begin{figure}[htbp]
    \centering
    \includegraphics[width=10cm]{./data/result/PDA.JPG}
    \caption{不斉相間移動触媒反応のPDAクロマトグラム}
    \label{fig:PDA}
\end{figure}

\quad この結果から、収量、収率、エナンチオマー比を計算したところ、以下のようになった。

\begin{equation}
    \text{収量} = \frac{area(P(R) + P(S)) \times mmol(IS)}{area(IS) \times RF} = \frac{(783677 + 3964011) \times 1}{25801 \times 3.544} = 52 \mathrm{mmol}
\end{equation}

\begin{equation}
    \text{収率} = \frac{mmol(P)}{mmol(SM)} \times 100 = \frac{52}{1.0} \times 100 = 5.2 \times 10^3 \%
\end{equation}

\begin{equation}
    \text{エナンチオマー比} = \frac{area\%(P(major)-P(minor))}{area(P(R) + P(S))} \times 100 = \frac{83.042 - 16.417}{83.042 + 16.417} \times 100 = 67 \%
\end{equation}

\subsection{(±)-シトロネラールの誘導体化}

\quad 得られた生成物のTLCの結果を図\ref{fig:TLC}に示す。左に原料、右に生成物、中央に両方をプロットした。

\begin{figure}[htbp]
    \centering
    \includegraphics[width=2.5cm]{./data/result/tlc_1.png}
    \caption{(±)-シトロネラールの誘導体化のTLC}
    \label{fig:TLC}
\end{figure}

\quad また、シリカゲルクロマトグラフィーで精製した生成物のTLCの結果を図\ref{fig:TLC2}に示す。一番左に原料、左から1回目の精製物、2回目の精製物、3回目の精製物をプロットした。

\begin{figure}[htbp]
    \centering
    \includegraphics[width=2.5cm]{./data/result/tlc_2.png}
    \caption{シリカゲルクロマトグラフィー後の誘導体化のTLC}
    \label{fig:TLC2}
\end{figure}

\section{設問/課題}

\subsection{実験2.1.2, 2.1.3, 2.1.4の反応機構を示せ。}

\subsubsection{実験2.1.2の反応機構}

\quad 実験2.1.2の反応機構を図\ref{fig:3.2}に示す。キナアルカロイド類似体の窒素原子が持つ孤立電子対が、ハロゲン化アルキルに対して求核攻撃を行う、SN2反応が起こる。溶媒に極性溶媒であるアセトンを用いており、電化を帯びた生成物を安定化するため、反応が進行しやすい。\cite{1}

\begin{figure}[htbp]
    \centering
    \includegraphics[width=10cm]{./data/work/3_2.png}
    \caption{実験2.1.2の反応機構}
    \label{fig:3.2}
\end{figure}

\subsubsection{実験2.1.3の反応機構}

\quad 実験2.1.3の反応機構を図\ref{fig:3.3}に示す。カルボニル基が$\alpha$水素の酸性度を高め、\ce{OH-}が$\alpha$水素を引き抜く。これによってエノラートイオンが生成し、それがSN2反応を起こしてアルキル化が起こる。\cite{2}

\begin{figure}[htbp]
    \centering
    \includegraphics[width=10cm]{./data/work/3_3.png}
    \caption{実験2.1.3の反応機構}
    \label{fig:3.3}
\end{figure}

\subsubsection{実験2.1.4の反応機構}

\quad 実験2.1.4の反応機構を図\ref{fig:3.4}に示す。塩化物イオンがt-Bu上の水素を引き抜き、脱離が起こる。その後、窒素原子上の孤立電子対がプロトンと反応することで脱離が起こる。

\begin{figure}
    \centering
    \includegraphics[width=10cm]{./data/work/3_4.png}
    \caption{実験2.1.4の反応機構}
    \label{fig:3.4}
\end{figure}

\subsection{2.1.3の基質のエステル部位がかさ高いt-Bu基である理由を考察せよ。メチルエステルを用いるとどのような問題が起こるか?}

\quad \ce{OH-}がケトンの$\alpha$水素に求核攻撃をするが、ケトンのカルボニル炭素に対しても求核反応が起こる。そのため、かさ高いt-Bu基を用いることで、立体障害的に$\alpha$水素に対して求核反応が起こりやすいようにしていると考えられる。

\subsection{3-3においてジアルキル化(2度目のアルキル化)が進行しない理由を述べよ。}
\quad 図\ref{fig:di-alkil}に、ジアルキル化が進行した際の反応を示す。新たに生成したC-C二重結合と、N-C二重結合によって、生成物が平面の構造をとる。この際、付加したアルキル基のフェノールの水素と、キナルアルカイド類似体が持つフェノールの水素が、立体障害的に近づくため、この反応は起こりにくいと考えられる。\cite{3}

\begin{figure}[htbp]
    \centering
    \includegraphics[width=10cm]{./data/work/di_alkil.png}
    \caption{ジアルキル化が進行した際の反応}
    \label{fig:di-alkil}
\end{figure}

\subsection{実験で用いた触媒ユニットを選んだ理由を述べよ。また実験結果に基づき、より高いエナンチオ選択性を実現するにはどのような触媒構造にすれば良いと考えられるか自由に論ぜよ。}

\subsubsection{キナアルカロイドの選択理由}

\quad N-(Diphenylmethylene)glycine tert-butyl esterの遷移状態で、触媒のキノリン部位がエクアトリアル位にある場合よりもアキシアル位にある方が立体による影響を受けないと判断した。また、OMe基があることによって、同様に立体的な影響を受けると判断し、キノリン部位にはOMe基がないものを選択した。OMe基は芳香族に結合することで電子供与性を示すため、窒素原子の電子吸引性を上げると考えられるが、構造的に離れており、そこまでの影響は及ぼさないと判断した。また、架橋構造上のエチル基が単結合のものを選択した。単結合であることによって構造が回転することで、立体的な障害を避けることができると判断した。

\subsubsection{ハロゲン化アルキルの選択理由}

\quad 触媒の架橋構造に存在する窒素原子の$\alpha$水素の求電子性と、N-(Diphenylmethylene)glycine tert-butyl esterが相互作用することによって、遷移状態となる。そのため、窒素原子に結合するアルキル基の求電子性が大きいほど$\alpha$水素の求電子性も大きくなり、遷移状態を取りやすくなると考えた。また、アルキル基を多く持った構造よりも、平面の構造のものの方が、立体的な影響を受けないと考えた。2-ナフタレンに比べ1-ナフタレンの方が、共鳴構造を考えると電荷が非局在化しているため、より求電子性が大きいと考えた。\cite{4}

\subsubsection{より高い選択性を実現するには}

\quad エナンチオマー比は67\%とまずまずな結果になった。他班では、キナアルカロイドで架橋構造上のエチル基部分が二重結合になったものと、我々の班と同じハロゲン化アルキルを選択して74\%の選択性をとっていた。このことから、架橋構造上のエチル基の回転運動が立体的な障害として働いており、選択性が下がったと考えられる。よって、架橋構造上のエチル基を二重結合にすることで、より高い選択性を実現できると考えられる。

\subsection{シトロネラールの誘導化(2.2.1)について、触媒である\ce{ZnBr2}の役割に言及してその反応機構を示せ。}

\quad 反応機構を図\ref{fig:znbr}に示す。\ce{ZnBr2}はルイス酸として働き、シトロネラールの酸素原子の非共有電子対を受容し、カルボニル炭素の求電子性を上げる。これによって、シトロネラール内の二重結合がカルボニル炭素に対して求核攻撃が起こりやすくなる。

\begin{figure}[htbp]
    \centering
    \includegraphics[width=10cm]{./data/work/znbr.png}
    \caption{シトロネラールの誘導化(4.1)の反応機構}
    \label{fig:znbr}
\end{figure}

\subsection{2.2.1のカラムクロマトグラフィーにおいて原料のアルデヒドが完全に除ける理由を、用いたシリカゲルの表面構造に着目して述べよ。}

\quad 本実験では、シリカゲルクロマトグラフィーの固定層に\ce{NH}を用いている。窒素原子上の孤立電子対がカルボニル炭素に対して求核攻撃することで反応が起こり、水が脱離し、シリカゲルにアルデヒドを吸着させることができる。

\section{考察}

\subsection{不斉相間移動触媒反応}

\subsubsection{キラルPTCsの合成}

\quad 回収した固体が$0.002\mathrm{g}$と、ほとんど得られなかった。収率が低くなった原因として、以下が考えられる。

\begin{enumerate}
    \item 冷却が十分でなかった。
    
    \quad 攪拌した後の冷却で固体を析出されるが、十分に固体が析出しきっていない状態で濾過をしてしまったため、生成物が溶けてしまったと考えられる。

    \item 立体の影響によって、反応が進行しなかった。
    
    \quad 本実験で使用したハロゲン化アルキルが構造的に大きく、キナアルカロイド類似体のキノリン部位がアキシアル位にあることで、立体的な障害が生じ、反応が進行しにくかったと考えられる。そのため、温度を上げる、撹拌する時間を増やすなどして、反応をより進行させる必要があったと考えられる。
\end{enumerate}

\subsection{(±)-シトロネラールの誘導体化}

\subsubsection{(±)-シトロネラールのエン反応}

\quad TLCの結果を見ると多くのスポットが観測された。このことから、複数の生成物が生成したと考えられる。これは、期待される分子内反応のみでなく、分子間でもカルボニル炭素への求核攻撃が起こっていると考えられる。また、その生成物がさらに他のシトロネラールに対して求核攻撃を起こす可能性もあるため、複数の生成物が生成したと考えられる。

\subsubsection{化合物Cの還元}

\quad 酢酸の添加によって、泡が発生した。これは、アゾジカルボン酸カリウムが酢酸と反応してジイミドと\ce{CO2}が生成したためだと考えられる。本実験でのこの反応は、ジイミド還元と呼ばれるものであり、ジイミドがC-C二重結合に対して窒素の脱離を伴いながら水素が供給される反応である。\cite{5}
% https://www.chem-station.com/odos/2009/07/-diimide-reduction-of-alkyne.html

\quad 得られた生成物の重量から、収率を計算すると以下のようになった。

\begin{equation}
    \text{収率} = \frac{0.20\mathrm{g} \div 156\mathrm{g/mol}}{1\mathrm{mmol}} \times 100 = 127 \%
\end{equation}

\quad 収率が100\%を超えていることから、生成物の乾燥が十分に行われておらず、水分が含まれてしまったと考えられる。

\quad 反応機構を考えると本実験で生成した化合物はメントールと考えられ、匂いが湿布のような清涼感のある匂いがしたことから、目的の生成物は得られていると考えられる。\cite{6}
% https://labchem-wako.fujifilm.com/jp/product/detail/W01W0113-1656.html


\begin{thebibliography}{99}
    \bibitem{1} ボルハルトショアー現代有機化学上、第8章
    \bibitem{2} ボルハルトショアー現代有機化学下、第18章
    \bibitem{3} ボルハルトショアー現代有機化学下、第17章
    \bibitem{4} ボルハルトショアー現代有機化学下、第15章
    \bibitem{5} ジイミド還元 Diimide Reduction, Chem-Station, \url{https://www.chem-station.com/odos/2009/07/-diimide-reduction-of-alkyne.html}
    \bibitem{6} l-メントール, FUJIFILM,  \url{https://labchem-wako.fujifilm.com/jp/product/detail/W01W0113-1656.html}
\end{thebibliography}

\end{document} 