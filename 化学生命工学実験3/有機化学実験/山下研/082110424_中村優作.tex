\documentclass{ltjsarticle}
\usepackage{amsmath}
\usepackage{amssymb}
\usepackage{graphicx} % Required for inserting images
\usepackage{enumerate}
\usepackage[version=4]{mhchem}
\usepackage{enumitem}
\usepackage{caption}
\usepackage{url}

\title{実験レポテンプレ}
\author{中村 優作}
\date{April 2023}

\pagestyle{plain}
\begin{document}
\flushleft{
    \huge{令和5年度 化学生命工学実験3}
    
    \vspace{100pt}
    
    \huge{エノラートアニオンによる炭素-炭素結合形成}
}

\vfill
\flushleft{
    \Large{\underline{班番号:8班}}
    
    \vspace{15pt}
    
    \Large{\underline{報告者:082110424 中村優作}}

    \vspace{15pt}

    \Large{\underline{共同実験者:中村⾥新}}

    \vspace{15pt}

    \Large{\underline{提出日:2023年11月23日}}

    \vspace{60pt}
}

\newpage

\section{目的}

\quad カルボニル基は、隣接する$\alpha$, $\beta$炭素原子に影響を与え、特に$\alpha$水素を酸性にする効果がある。結果として、エノラートアニオンを形成し、エノラートアニオンを利用することで、新しく炭素-炭素結合を合成することが可能である。本実験では、アセトンとベンズアルデヒドからジベンジリデンアセトンを合成する。また、2,4-ジニトロフェニルヒドラジンとフェニルヒドラジンをジベンジリデンアセトンと反応させ、得られた生成物の違いについて考察することを目的とする。


\section{実験操作・結果}

\subsection{ジベンジリデンアセトンの調整}

\quad $100\mathrm{mL}$丸底フラスコに、\ce{NaOH}$2.1\mathrm{g}$、水$25\mathrm{mL}$を加え、\ce{NaOH}が溶けるまで撹拌した。エタノール$20\mathrm{mL}$を加えて攪拌を続けた。別のフラスコに、ベンズアルデヒド$2.6\mathrm{g}$とアセトン$0.73\mathrm{g}$の混合溶液を調整した。この溶液の半分を撹拌中の溶液に加えたところ、黄色の溶液が得られた。その後残りの溶液を全て加え、容器をエタノールで洗って合わせ、30分間攪拌を続けた。溶液を吸引濾過し、塩基を除去するために水で洗い、濾紙で挟んで乾燥させた。得られた個体の重量を測定したところ、$2.7\mathrm{g}$であった。酢酸エチルを$6.6\mathrm{mL}$加えて個体を$80\mathrm{^{\circ}C}$程度の湯浴で溶かした。この際、溶媒が少し沸騰してしまった。氷浴槽で冷やして再結晶を行い、吸引濾過で個体を得た。得られた個体の重量を測定したところ、$1.1\mathrm{g}$であった。また、融点を測定したところ、$107$〜$110\mathrm{^{\circ}C}$であった。

\quad 理論値を計算すると以下のようになった。

\begin{equation}
    \mathrm{Yield} = \frac{2.6\mathrm{g}}{106\mathrm{g/mol}} \times \frac{1}{2} \times 234\mathrm{g/mol} = 2.9\mathrm{g}
\end{equation}

したがって、粗生成物時の収率は$2.66\mathrm{g}/2.87\mathrm{g} \times 100 = 92\%$、純生成物時の収率は$1.12\mathrm{g}/2.87\mathrm{g} \times 100 = 39\%$であった。

\subsection{ジベンジリデンアセトンと2,4-ジニトロフェニルヒドラジンの合成}

\quad 本操作では、純粋なエタノールと誤って70\%エタノールを用いている。$50\mathrm{mL}$三角フラスコに、2,4-ジニトロフェニルヒドラジンジニトロフェニルヒドラジン$0.19\mathrm{g}$、70\%エタノール$15\mathrm{mL}$を加えたところ、オレンジ色の溶液が調整された。$80\mathrm{^{\circ}C}$水浴で5分間撹拌した。別のフラスコに、ジベンジリデンアセトン$0.24\mathrm{g}$、70\%エタノール$5\mathrm{mL}$を混合したところ、黄色の溶液が得られた。この溶液を撹拌中の溶液に加えた。容器を70\%エタノールで洗浄して加えた。パラトルエンスルホン酸$14\mathrm{mg}$を加え、$80\mathrm{^{\circ}C}$で30分ほど温めたところ深赤色の固体が析出した。水浴で室温まで冷却し、吸引濾過で赤色の固体を得た。容器を$10\mathrm{mL}$の70\%エタノールで洗浄して加えた。得られた個体の重量を測定したところ、$0.092\mathrm{g}$であった。$5\mathrm{mL}$の酢酸エチルに溶解させ、$80\mathrm{^{\circ}C}$の湯浴で溶かした。$5\mathrm{mL}$のヘキサンを加え、一晩室温で静置した。濾過をし、得られた固体の重量を測定したところ、$0.035\mathrm{g}$であった。融点を測定したところ、$181$〜$185\mathrm{^{\circ}C}$であった。

\quad 理論値を計算すると以下のようになった。

\begin{equation}
    \mathrm{Yield} = \frac{0.19\mathrm{g}}{198\mathrm{g/mol}} \times 414\mathrm{g/mol} = 0.40\mathrm{g}
\end{equation}

したがって、粗生成物時の収率は$0.092\mathrm{g}/0.397\mathrm{g} \times 100 = 23\%$、純生成物時の収率は$0.035\mathrm{g}/0.397\mathrm{g} \times 100 = 8.8\%$であった。

\subsection{フェニルヒドラジンとジベンジリデンアセトンの反応}

\quad $50\mathrm{mL}$三角フラスコに、ジベンジリデンアセトン$0.33\mathrm{g}$、パラトルエンスルホン酸$39\mathrm{mg}$、フェニルヒドラジン$0.38\mathrm{g}$、エタノール$10\mathrm{mL}$を加えた。この際、溶液は黄色透明の液体であった。$80\mathrm{^{\circ}C}$で30分撹拌したところ、黄色に濁った溶液が得られた。氷浴槽で結晶を析出させ、吸引濾過で個体を得た。この際、濾液が再結晶を続けてしまっていた。容器を$10\mathrm{mL}$の70\%エタノールで洗浄して加えた。得られた個体の重量を測定したところ、$0.13\mathrm{g}$であった。$80\mathrm{^{\circ}C}$で$6\mathrm{mL}$のエタノールに溶かし、$3\mathrm{mL}$のヘキサンを加えた。氷水浴で再結晶させ、吸引濾過をして個体を得た。重量を測定したところ、$0.038\mathrm{g}$、融点を測定したところ、$147$〜$150\mathrm{^{\circ}C}$であった。

\quad 理論値を計算すると以下のようになった。

\begin{equation}
    \mathrm{Yield} = \frac{0.33\mathrm{g}}{234\mathrm{g/mol}} \times 324\mathrm{g/mol} = 0.46\mathrm{g}
\end{equation}

したがって、粗生成物時の収率は$0.134\mathrm{g}/0.457\mathrm{g} \times 100 = 29\%$、純生成物時の収率は$0.038\mathrm{g}/0.457\mathrm{g} \times 100 = 8.3\%$であった。

\section{課題}

\subsection{Write a stepwise mechanism of the formation of benzalacetone.}
% ベンズアラセトンの生成メカニズムを段階的に書きなさい。

\quad 反応機構を図\ref{fig:3.1}に示す。この反応では、アルドール縮合と呼ばれる反応が起こっている。ケトンの$\alpha$水素は酸性であり、強い塩基である\ce{OH-}によって引き抜かれ、エノラートイオンを生成する。エノラートイオンが、アルデヒドの求電子性のカルボニル炭素に対して求核攻撃を行うことで、炭素-炭素結合が新しく形成される。水酸化物イオンは優れた脱離基ではないが、最終生成物は共役により安定化された生成物が生じ、熱力学的に有利になるため、水酸化物イオンの脱離が進む。\cite{1}

\begin{figure}[htbp]
    \centering
    \includegraphics[width=10cm]{./data/3.1.png}
    \caption{ベンズアラセトンの合成の反応機構}
    \label{fig:3.1}
\end{figure}

\subsection{Assign the 1H NMR spectrum of 1,5-diphenyl-3-styrylpyrazoline (Fig. 2) formed from the reaction of dibenzalacetone with phenylhydrazine. Particular attention should be paid to explain the coupling pattern of signals around d 2.9 – 5.4}

% ジベンズアセトンとフェニルヒドラジンの反応から生成した1,5-ジフェニル-3-スチリルピラゾリン(図2)の1 H NMRスペクトルを 解析する。特にδ2.9〜5.4付近のシグナルのカップリングパターンの説明に注意すること。

\quad 1,5-ジフェニル-3-スチリルピラゾリンの1H NMRスペクトルを図\ref{fig:nmr}に示す。また、1,5-ジフェニル-3-スチリルピラゾリンの構造を図\ref{fig:nmr_structure}に示す。

\begin{figure}[htbp]
    \centering
    \includegraphics[width=15cm]{./data/nmr.png}
    \caption{1,5-ジフェニル-3-スチリルピラゾリンの1H NMRスペクトル}
    \label{fig:nmr}
\end{figure}

\begin{figure}[htbp]
    \centering
    \includegraphics[width=8cm]{./data/nmr_structure.png}
    \caption{1,5-ジフェニル-3-スチリルピラゾリンの構造}
    \label{fig:nmr_structure}
\end{figure}

\quad 次のように帰属をした。

\begin{enumerate}[label=(\alph*)]
    \item 1H, 3.05 ppm, dd
    \item 1H, 3.76 ppm, dd
    \item 1H, 5.30 ppm, dd
    \item 1H, 6.65 ppm, d
    \item 1H, 6.80 ppm, d
    \item 2H, 6.95 ppm, m
    \item 13H, 7.20 ppm, m
\end{enumerate}

\quad a, bは同じ炭素についているが、環で固定されているため非等価になる。そのためジェミナルカップリングが起こり、a, b, cの水素でddにピークが分裂する。cの水素は窒素原子によって低磁場シフトしているため、5.30ppmのピークと考えられ、a,bは3.05ppm, 3.76ppmのピークと考えられる。aとbでは、bが近くにあるフェニル基の反遮蔽空間にあるため、低磁場にシフトすると考えられ、bが3.76ppm, aが3.05ppmのピークだと考えられる。d,eは、アルケンにより低磁場シフトしたピークだと考えられる。フェニル基が近くにあるeが、異方性効果を受けて、より低磁場にシフトしていると考えられる。また、fはNによって電化が+に偏った炭素に隣接しているため、他のベンゼン環上のH(g)に比べて高磁場にシフトしていると考えられる。


\subsection{Write a stepwise mechanism of the formation of 1,5-diphenyl-3-styrylpyrazoline and explain the reason why products are different when dibenzalacetone was treated with phenylhydrazine and 2,4-dinitrophenylhydrazine.}
% 1,5-ジフェニル-3-スチリルピラゾリン生成の段階的メカニズムを書き、ジベンザラセトンをフェニルヒドラジンと2,4-ジニトロフェニルヒドラジンで処理したときに生成物が異なる理由を説明しなさい。

\quad 反応機構を図\ref{fig:3.3}に示す。この反応機構は酸性条件下での反応である。塩基性の大きい窒素原子が、電子不足のカルボニル炭素に求核攻撃することで反応が始まる。図\ref{fig:3.3}に示した共鳴構造を経て、5員環が生成すると考えられる。これは、構造的に5員環の形をとっているため、このような環を形成する反応が進行しやすいためだと考えられる。2,4-ジニトロフェニルヒドラジンで処理した時は、分子内水素結合により、ヒドラゾンのN-Hと、オルトニトロ基酸素の間に、6員平面環が形成される。この環は、5員環よりも安定なため、この反応が進行すると考えられる。\cite{2}

\begin{figure}[htbp]
    \centering
    \includegraphics[width=12cm]{./data/3.3.png}
    \caption{1,5-ジフェニル-3-スチリルピラゾリンの合成の反応機構}
    \label{fig:3.3}
\end{figure}

\section{結論及び考察}

\subsection{ジベンジリデンアセトンの調整}

\quad ジベンジリデンアセトンの融点は$104$〜$107\mathrm{^{\circ}C}$であり、\cite{3}得られた生成物の融点の測定値は$107$〜$110\mathrm{^{\circ}C}$であった。このことから、目的生成物であるジベンジリデンアセトンを得ることが出来たと言える。純生成物の収率は38\%と低くなった。これは、再結晶の際に加えた酢酸エチルが多く、完全には再結晶による析出が行われなかったことや、再結晶の際に溶媒が沸騰してしまったことが原因と考えられる。これらの再結晶の際の問題を解決することで、より収率を上げることが出来たと考えられる。

\subsubsection{より収率を上げるには}

\quad アセトンのようなケトンでは、アルドール付加の駆動力が弱い。これは、ケトンがアルデヒドに比べて安定であるからである。アルドール付加をより前進させるためには、生成物であるアルコールを反応混合物から取り除くことや、高温下などで反応を行い、脱水反応を起こして水を除去して平衡を生成物側に偏らせることが考えられる。\cite{1}

\subsection{ジベンジリデンアセトンと2,4-ジニトロフェニルヒドラジンの合成}

\quad ジベンジリデンアセトンと2,4-ジニトロフェニルヒドラジンから、赤色の固体を合成することが出来た。本操作では誤って70\%エタノールを使用したため、固体が水に溶けて収率が極端に低くなってしまったと考えられる。

\subsubsection{パラトルエンスルホン酸を加えた理由}

\quad 本実験の反応では、酸性条件下の方が反応が進行しやすい。これは、ヒドロキシ基が脱離する反応が速くなるためである。このため、酸触媒としてパラトルエンスルホン酸を加えた。この反応では有機溶媒を用いているため、\ce{HCl}などの酸触媒を用いることが難しい。また、\ce{HCl}などの酸触媒を用いると、生成物が塩酸塩となってしまうため、再結晶が難しくなる。このため、パラトルエンスルホン酸を用いたと考えられる。

\subsection{フェニルヒドラジンとジベンジリデンアセトンの反応}

\quad フェニルヒドラジンとジベンジリデンアセトンから、黄色の結晶を合成することが出来た。吸引濾過での濾液で再結晶が進行していたため、収率が低くなったと考えられる。十分な時間をかけて再結晶を行うことで、より収率を上げることが出来たと考えられる。

\subsubsection{5員環が安定な理由}

\quad 今回の合成で生成される5員環は、ピラゾール骨格と呼ばれる、3つの炭素原子と2つの隣接する窒素原子より構成される5員複素環化合物である。ピラゾール骨格では、図\ref{fig:4.3.1}のような共鳴構造を取る\cite{4}ため、共鳴による安定化を受ける。したがって、本実験での最終生成物として、1,5-diphenyl-3-styrylpyrazolineが得られたと考えられる。

\begin{figure}[htbp]
    \centering
    \includegraphics[width=10cm]{./data/4.3.1.png}
    \caption{ピラゾール骨格の共鳴構造}
    \label{fig:4.3.1}
\end{figure}

\begin{thebibliography}{99}
    \bibitem{1} ボルハルト・ショアー現代有機化学第8版下, 18章
    \bibitem{2} Crystal Structure and Hirshfeld Surface Analysis of 1,4-Pentadien-3-one, (1E,4E)-1,5-diphenyl-2-(2,4-dinitrophenyl)hydrazone \url{https://www.researchgate.net/publication/349211475_Crystal_Structure_and_Hirshfeld_Surface_Analysis_of_14-Pentadien-3-one_1E4E-15-diphenyl-2-24-dinitrophenylhydrazone}
    \bibitem{3} trans,trans-ジベンジリデンアセトン, FUJIFILM, \url{https://labchem-wako.fujifilm.com/jp/product/detail/W01W0232-8056.html}
    \bibitem{4} Styrylpyrazoles: Properties, Synthesis and Transformations, National Library of Medicine, \url{https://www.ncbi.nlm.nih.gov/pmc/articles/PMC7764498/}
\end{thebibliography}

\section{感想等}

\quad それぞれの操作で、目的の生成物を合成することができた。誤って70\%エタノールを使用してしまったり、再結晶が十分でないなど、簡単なミスによって収率が低くなってしまっため、今後はより注意して実験を行いたいと思った。

\end{document} 