\documentclass{ltjsarticle}
\usepackage{amsmath}
\usepackage{amssymb}
\usepackage{graphicx} % Required for inserting images
\usepackage{enumerate}
\usepackage[version=4]{mhchem}
\usepackage{caption}
\usepackage{url}

\title{実験レポテンプレ}
\author{中村 優作}
\date{April 2023}

\pagestyle{plain}
\begin{document}
\flushleft{
    \huge{令和5年度 化学生命工学実験3}
    
    \vspace{100pt}
    
    \huge{遺伝子組み換え実験}
}

\vfill
\begin{flushright}

        \Large{\underline{学籍番号 : 082110424}}

        \vspace{30pt}
    
        \Large{\underline{氏名 : 中村優作}}

        \vspace{30pt}
    
        \Large{\underline{実施日 : 10/18, 19}}
        
\end{flushright}

\newpage

\section{目的}

\quad 遺伝子組み換えの技術は疾病の遺伝子治療や動物のクローン作製などに応用され、大きな成果を挙げつつある。本実験では、遺伝子の大腸菌への導入、連結酵素によるDNAの結合、制限酵素によるDNA断片の切断を行い、得られたDNA断片を電気泳動により分析することを目的とする。

\section{操作}

\subsection{PCR}

\quad PCR酵素$15\mathrm{\mu L}$が入ったマイクロチューブに超純水$3\mathrm{\mu L}$とtemplate$6\mathrm{\mu L}$とprimer$6\mathrm{\mu L}$を加えて調整した。調整した溶液を装置にセットし、以下の図\ref{fig:pcr}プログラムでPCRを行った。

\begin{figure}[htbp]
    \centering
    \includegraphics[width=12cm]{./data/pcr.png}
    \caption{PCRプログラム}
    \label{fig:pcr}
\end{figure}

% TODO:原理

\subsection{電気泳動}

\quad 6x loading buffer入りのエッペンチューブにPCR後の溶液を$10\mathrm{\mu L}$加えた。その後$100\mathrm{V}$で20分間電気泳動をした。その後、電気泳動で確認できたバンドを切り出してカラムでDNAを観察した。ゲル精製したEGFPと$GABA_{B1}$をコードするDNA断片とベクターとなるpBlueScriptを制限酵素で切断した。

\subsection{ligation}

\quad エッペンチューブ2本に制限酵素処理をしたpBluescript(vector)を$2.5\mathrm{\mu L}$ずつ加え、続いて片方にEGFPをコードするDNA断片(insert)、もう片方に$GABA_{B1}$をコードするDNA断片(insert)を$2.5\mathrm{\mu L}$加えた。酵素と緩衝液であるligation mixを$5\mathrm{\mu L}$加えて、$16\mathrm{℃}$で30分反応させた。$2\mathrm{\mu L}$の6x loading bufferが入ったエッペンチューブにligation前のペリレン、EGFP, GABAを各$3\mathrm{\mu L}$ずつ加えた。また、$1\mathrm{\mu L}$の6x loading bufferが入ったエッペンチューブ2本にそれぞれペリレンとEGFP$4\mathrm{\mu L}$ずつ、ペリレンとGABA$4\mathrm{\mu L}$ずつ加えた。これらの溶液を$16\mathrm{^{\circ}C}$で30分反応させた。その後、30分間$100\mathrm{V}$で電気泳動をした。

\subsection{transformation}

\quad 2.3で調整したペリレンとEGFPの溶液、ペリレンとGABAの溶液を$5\mathrm{\mu L}$とり、新しいチューブ内でコンピテントセル$30\mathrm{\mu L}$とそれぞれの溶液を混合した。タッピング後、氷上で5分間静置して、コンピテントセルを$42\mathrm{^{\circ}C}$で45分間加熱した。その後5分間静置してLBプレートへ菌を撒いた。LBプレートに生えたコロニーを1個突き、液体培地$2\mathrm{mL}$に加えた。

\subsection{ミニプレップ}

\quad エッペンチューブに菌体を移し、$4000\mathrm{rpm}$で1分間遠心した。上澄みを除去して、Solution1($50\mathrm{\mu M}$ グルコース, $20\mathrm{mM}$ Tris-\ce{HCl}(pH8.0), $10\mathrm{mM}$ EDTA)を$100\mathrm{mL}$加えてボルテックスを行った。次に、Solution2($0.2\mathrm{M}$ \ce{NaOH}, 1\% SDS)を$200\mathrm{\mu L}$加えて転倒攪拌した。Solution3($2\mathrm{M}$ 酢酸, $3\mathrm{M}$酢酸カリウム)を$150\mathrm{\mu L}$を加えて転倒攪拌した。この溶液を5分間$1200\mathrm{rpm}$で遠心した。その後、新しいチューブに上澄みを移し、イソプロパノールを$400\mathrm{\mu L}$加えてボルテックスした。再び5分間$1200\mathrm{rpm}$で遠心した。溶液の上澄みを除去し、70\%エタノール$1\mathrm{mL}$を加えた。再び5分間$1200\mathrm{rpm}$で遠心した。上澄みを完全に除去し、$55\mathrm{^{\circ}C}$で乾燥した。その後MilliQ$30\mathrm{\mu L}$によりDNAを溶解させた。

\subsection{制限酵素処理によるDNA断片の確認}

\quad DNA溶液$4\mathrm{\mu L}$とmilliQ$11\mathrm{\mu L}$、10x制限酵素緩衝液(rCutSmart)$2\mathrm{\mu L}$を制限酵素入りの溶液$3\mathrm{\mu L}$に混ぜた。その後、$37\mathrm{^{\circ}C}$30分。反応後、反応溶液$5\mathrm{\mu L}$を$1\mathrm{\mu L}$の6x loading bufferが入ったエッペンチューブに加えてアガロースゲル電気泳動を$100\mathrm{V}$で30分間行った。
\section{原理}

\subsection{カラム精製}

\quad 固相抽出技術を利用し、シリカまたはガラスファイバーのフィルターメンブレンにDNAを結合させ、精製/分離を行う方法である。遠心処理を行うことで、DNAのシリカへの結合、洗浄、溶出を行うことができる。\cite{1}

\subsection{制限酵素処理}

\quad 制限酵素は、二本鎖DNAの特定の塩基配列を認識して結合し、切断する酵素である。制限酵素は切断に必要な因子や切断の仕方によってⅠ型からⅢ型の3種類に分けられる。Ⅰ型, Ⅲ型は切断部位が一定でないのに対し、Ⅱ型は特定の部位(通常DNA上の回転対称配列)を切断するため、遺伝子組み換え実験にはⅡ型酵素が用いられる。\cite{2}

\subsection{Ligation}

\quad ライゲーションとは、DNAリガーゼを使ってDNA断片同士をつなぐ反応で、DNA末端を5'リン酸と3' OH基間で結合するものである。\cite{3}

\subsection{Loading buffer}

\quad Loading bufferは、電気泳動用核酸サンプルを調整する際に添加する試薬である。ローディングバッファーにはアプライの際にウェルの中にDNAサンプルが沈むよう比重を増やすためのグリセロールまたはフィコールと、電気泳動の進行を観察できる色素が含まれている。\cite{4}

\subsection{電気泳動}

\quad ゲル電気泳動は、拡散やタンパク質等の生体分子を高分子ハイドロゲル中において電気泳動し、分子サイズに移動した泳動度の違いを利用して分離、分析する方法である。殿下を有する分子は溶液中において、伝場中に置かれると一定方向に向かって動き出す。この時、不電化を有する分子は陽極側に、正電荷を有する分子は陰極側に泳動される。高分子ハイドロゲル中において、これらの分子が泳動された時、網目からの抵抗を受けることで分子はそのゲルサイズに依存してゲル中を泳動することが知られている。大きな分子ほど遅く、小さな分子ほど早く泳動される。これは高分子ゲルが作り上げる網目の間を分子が通り抜ける速度が分子サイズに依存しているためである。\cite{5}この性質を利用して、DNAやタンパク質等の生体高分子を分離、分析することができる。

\subsection{EtBr}

\quad EtBr(臭化エチジウム)は、DNA塩基対に似た挿入剤であり、核酸蛍光タグとして使用することができる。EtBrはDNAと結合すると強い蛍光を発する。これは、塩基対の間に見られる疎水性環境のためである。この環境に移動し、溶媒から離れることにより、EtBrカチオンは水に関連する分子を放出させられる。水は強力な蛍光消光剤であるため、水分子を除去するとEtBrが蛍光を発するようになる。\cite{6}

\subsection{形質転換}

\quad 形質転換(transformation)は、外来性DNAが宿主細胞に移入されるプロセスである。形質転換の方法には、ケミカルトランスフォーメーション、エレクトロポレーションなどがある。ケミカルトランスフォーメーションでは、塩化カルシウムのような2価の陽イオンで処理することにより、コンピテントセル(細胞が外来DNAを取るこむことができる状態)にし、最近の細胞壁をDNAに対して透過性にする。熱ショックによって細胞膜に一時的に孔を形成し、外来DNAの細胞内への移行を可能にする。エレクトロポレーションでは短い電気パルスを用いて最近細胞を一時的に投下性にする。この状態で外来DNAを細胞に加えることで、細胞内に取り込まれる。\cite{7}

\subsection{アンピシリン(Amp)}

\quad アンピシリンはペニシリン系の抗生物質で、バクテリア細胞壁のペプチドグリカン架橋の合成阻害することでバクテリアの生育を阻害する。D-アラニル-D-アラニンに作用し、ペプチドグリカン合成に関与するトランスペプチダーゼが類似体であるアンピシリンを誤認することで細胞壁の架橋が阻害される。このため菌の細胞壁は細胞分裂ごとに弱くなり、数回の細胞分裂の末、浸透圧に耐えられなくなり溶菌して死滅する。一方で、アンピシリン分解酵素であるβラクタマーゼはアンピシリンのβラクタム構造を加水分解することでアンピシリンを失活させる。目的遺伝子をプラスミドベクターに組込み形質転換を行うとき、ベクターDNAにβラクタマーゼ遺伝子が含まれていると、アンピシリン存在下で培養した場合、プラスミドDNAが導入されたバクテリアが生育でき、選び出すことができる。\cite{8}

\subsection{IPTG}

\quad IPTG(isopropyl β-D-thiogalactopyranoside)は、大腸菌(E.coli)などのラクトース(lac)オペロン制御下にある遺伝子の発現を誘導する化合物である。IPTGは、lacリプレッサーに結合し不活性化することでlacオペロンの転写を誘導する機能を持ち、また発現するβ-ガラクトシダーゼ(lacZ)の基質とはならないため、持続的に機能する。\cite{9}


\subsection{ミニプレップ}

\quad ミニプレップは、プラスミドDNAを抽出するための手法である。Solution1に含まれるリゾチームは、細胞壁の分解、グルコースは、浸透圧を与えて細胞膜を破壊する役目、EDTAは金属イオンを活性中心に持つデオキシリボヌクレアーゼを阻害する目的がある。この操作は細胞壁を溶解する操作である。Solution2は、界面活性剤とpHにより、細胞成分を完全に溶解するのがこの操作である。SDSは細胞の膜構造を破壊し、水酸化ナトリウムの高いpHにより染色体DNAを変性させる。Solution3はpHを急激に中性に戻すことで、溶解したタンパク質とSDS,染色体DNAが複合体を形成し、不溶性の凝集物を形成する。\cite{10}

\subsection{アルコール沈殿}

\quad アルコール沈澱は、DNAを溶液から沈殿させる方法である。DNAは負に帯電したリン酸骨格を持ち、水溶液中で親水性コロイドとして存在している。エタノールなどの水溶性の有機溶媒を加えると水和水が奪われる。これに塩を加えるとリン酸基の負電荷間の反発が解消され、ファンデルワールス力が優勢となり、DNA鎖どうしが凝集して沈澱する。磯プロパノールはエタノールより疎水性が高いため、核酸の溶解度が低く、より少ない量で拡散を沈澱させることができる。\cite{11}

\section{結果}

\subsection{PCR後の電気泳動}

\quad PCR後の電気泳動の結果を図\ref{fig:pcr_result}に示す。この段階ではバンドを確認することができなかった。

\begin{figure}[htbp]
    \centering
    \includegraphics[width=5cm]{./data/pcr_result.png}
    \caption{PCR後の電気泳動(左から4レーン目)}
    \label{fig:pcr_result}
\end{figure}

\subsection{Ligationの電気泳動}

\quad Ligation後の電気泳動の結果を図\ref{fig:ligation_result_before}に示した。

% TODO: 検量線を作成してDNA断片の大きさを求める

\begin{figure}[htbp]
    \centering
    \includegraphics[width=12cm]{./data/ligation.png}
    \caption{Ligation後の電気泳動(2-2)}
    \label{fig:ligation_result_before}
\end{figure}

\subsection{Transformation}

\quad 今回の実験では、Transformation後のLBプレートにコロニーが発生しなかった。

\subsection{制限酵素処理}

\quad 制限酵素処理後の電気泳動の結果を図\ref{fig:restriction_result}に示す。

% TODO: なんか班ごとに入ってるやつと入ってないやつがあったから聞くこと
% TODO: 検量線を作成してDNA断片の大きさを求める

\begin{figure}[htbp]
    \centering
    \includegraphics[width=12cm]{./data/restriction.png}
    \caption{制限処理後の電気泳動(2-2)}
    \label{fig:restriction_result}
\end{figure}

\section{設問/課題}

\subsection{PCR法で遺伝子増幅を行う際、増幅反応が理想的に進行した場合、8回のサイクル反応で、目的遺伝子は何倍に増幅されるか。}

\quad 1サイクルで2倍に増幅されるため、$2^{8}=256$倍に増幅される。これは副生成物を含めた全てのDNAを表している。副生成物はn回のサイクルで2nずつ増幅されるため、目的遺伝子は$2^8 - 2\times8 = 240倍$となる。

\subsection{PCR法における反応の理想条件について論ぜよ。}

\quad PCRに影響を与える因子には以下のものがある。\cite{12}

\begin{enumerate}
    \item 酵素濃度
    
    Tag DNAポリメラーゼは、通常$1〜2.5\mathrm{units/\mu L}$反応液で使用される。一般に濃度が高すぎると非特異的産物が出現することがあり、逆に濃度が低すぎると増幅が不十分になる。
    \item dNTP濃度
    
    通常それぞれ$20〜200\mathrm{\mu mol/l}$である。各dNTPの濃度は同一でないと誤った取り込みによるエラーが生じる原因になる。

    \item \ce{Mg^{2+}}濃度
    
    通常、総dNTP濃度より$0.5〜2.5\mathrm{mmol/l}$高い濃度が採用される。鋳型DNAやプライマーから持ち込まれるEDTAなどのキレート剤の濃度に注意する必要がある。
    \item プライマー
    
    プライマーは通常18〜28ヌクレオチド長で、G+C含量が50〜60\%で、Tm値が$55〜80\mathrm{^{\circ}C}$となるように設計する。2つのプライマーはそれぞれ$0.1〜0.5\mathrm{\mu mol/l}$で使用する。

    \item 温度
    
    変性温度は高いほど特異性や増幅度が高くなるが、逆にTagDNAポリメラーゼの活性低下を早めるので、普通は$94〜96\mathrm{^{\circ}C}$で15〜30秒間である。プライマーのアニーリング温度は、Tm値より$5\mathrm{^{\circ}C}$程低い温度が良く、通常$50〜60\mathrm{^{\circ}C}$である。アニーリング温度が高いほど特異性は高くなる。伸長反応の温度は、$72\mathrm{^{\circ}C}$が多く用いられる。

    \item サイクル数
    
    サイクル数は多いほど増幅度は高くなるが、同時に非特異的産物も増加する。通常は40サイクルを超えないことが望ましい。
\end{enumerate}

\subsection{Ligation溶液(P+EまたはP+G)を加えて形質転換させた大腸菌を播いたLBプレート上の青と白のコロニーの比率を比較し、違いがある場合にはその原因を考察せよ。}

\quad 本実験では、コロニーが発生しなかったため、他の班のデータを使用する。LBプレートに生えたコロニーを図\ref{fig:transformation_E}, \ref{fig:transformation_G}に示す。

\begin{figure}[htbp]
    \begin{minipage}{0.5\hsize}
        \centering
        \includegraphics[width=6cm]{./data/transformation_E.png}
        \caption{LBプレート(ペリレンとEGFP)の結果(2-2)}
        \label{fig:transformation_E}
    \end{minipage}
    \begin{minipage}{0.5\hsize}
        \centering
        \includegraphics[width=6cm]{./data/transformation_G.png}
        \caption{LBプレート(ペリレンとGABA)の結果(2-2)}
        \label{fig:transformation_G}
    \end{minipage}
\end{figure}

それぞれのコロニーの数を数えた結果を表\ref{tab:transformation_E}, \ref{tab:transformation_G}に示す。

\begin{table}[htbp]
    \centering
    \caption{LBプレート(ペリレンとEGFP)の結果(2-2)}
    \begin{tabular}{|c|c|c|c|} \hline
        青 & 白 & 合計 & 青/白 \\ \hline
        2 & 135 & 137 & 0.015  \\ \hline
    \end{tabular}
    \label{tab:transformation_E}
\end{table}

\begin{table}[htbp]
    \centering
    \caption{LBプレート(ペリレンとGABA)の結果(2-2)}
    \begin{tabular}{|c|c|c|c|} \hline
        青 & 白 & 合計 & 青/白 \\ \hline
        0 & 8 & 8 & 0 \\ \hline
    \end{tabular}
    \label{tab:transformation_G}
\end{table}

結果から、ペリレンとEGFPを用いた場合は青いコロニーが残存したが、ペリレンとGABAを用いた場合は、青いコロニーが残存しなかった。これは、ペリレンとEGFPを用いた場合は、完全にはすべてのプラスミドに外来DNAが挿入されていなかったため、青いコロニーが残存したと考えられる。一方で、ペリレンとGABAを用いた場合は、完全にはすべてのプラスミドに外来DNAが挿入されていたため、青いコロニーが残存しなかったと考えられる。

% \quad また、ペリレンとEGFPを用いた場合の方が、発生したコロニーの数が多かった。
% TODO

\subsection{青いコロニーと白いコロニーが生成するのはなぜか?}

\quad \beta-D-ガラクトシダーゼの基質である青色色素X-galを寒天培地に加えておくと、\beta-D-ガラクトシダーゼの作用により分解されてコロニーは青色になる。一方で、この酵素活性を欠くコロニーは白色になる。このようにして、外来DNAがプラスミドに挿入されたかどうかを判断することができる。\cite{13}
% https://www.sbj.or.jp/wp-content/uploads/file/sbj/9401/9401_yomoyama.pdf

\subsection{電気泳動によるDNAの確認の際に、白いコロニーを液体培養したにもかかわらず制限酵素反応から想定されるバンドが検出できないことがある。その原因を推測せよ。}

\quad 制限酵素の中には、DNAに強く結合しているものも存在する。そのため、想定されるバンドより大きいものまたはスメアが検出されることが考えられる。この場合、SDSを含むローディング色素を使用し、切断されたDNAから酵素を加熱して分離することによって誤ったゲルシフトを防ぐことができる。\cite{14}
% https://www.thermofisher.com/jp/ja/home/life-science/cloning/cloning-learning-center/invitrogen-school-of-molecular-biology/molecular-cloning/restriction-enzymes/restriction-enzyme-key-considerations.html

\subsection{制限酵素反応後の電気泳動で確認されたバンドが示すDNAのサイズを決定せよ。}

\quad 図\ref{fig:restriction_result}から、検量線を作成し、観測されたバンドから確認されたDNA断片の大きさを求めた。検量線を図\ref{fig:line_restriction}、各バンドの泳動距離を図\ref{tab:restriction}に、DNA断片を求めた結果を表\ref{tab:restriction_result}に示す。

\begin{figure}[htbp]
    \centering
    \includegraphics[width=8cm]{./data/line_restriction.png}
    \caption{制限処理後の電気泳動の検量線}
    \label{fig:line_restriction}
\end{figure}

\begin{table}[htbp]
    \centering
    \caption{制限酵素反応後の電気泳動で確認されたバンドの泳動距離(px)}
    \begin{tabular}{|c|c|c|} \hline
        EGFP(EcoRl, XhoⅠ) & GABA(EcoRl, XhoⅠ) & GABA(EcoRl, HindⅢ) \\ \hline
        328 & 297 & 327 \\
        452 & 324 & 360 \\
        & & 412 \\ \hline
    \end{tabular}
    \label{tab:restriction}
\end{table}

\begin{table}[htbp]
    \centering
    \caption{制限酵素反応後の電気泳動で確認されたバンドが示すDNAのサイズ(bp)}
    \begin{tabular}{|c|c|c|} \hline
        EGFP(EcoRl, XhoⅠ) & GABA(EcoRl, XhoⅠ) & GABA(EcoRl, HindⅢ) \\ \hline
        2926 & 3811 & 2951 \\
        1017 & 3028 & 2228 \\
        & & 1431 \\ \hline
    \end{tabular}
    \label{tab:restriction_result}
\end{table}

また、制限酵素処理で理論的に確認されるバンドについて図\ref{fig:restriction_theory}に示す。

\begin{figure}[htbp]
    \centering
    \includegraphics[width=12cm]{./data/logic_restriction.png}
    \caption{制限酵素反応後の電気泳動で確認されるバンドの理論値}
    \label{fig:restriction_theory}
\end{figure}

実測値と比較すると、EGFPでのサイズが小さいバンド以外での誤差が±5\%以内であり、制限酵素処理によってておおむね正しくDNA断片が得られたと考えられる。EGFPのDNAが小さい方のバンドで理論値と実測値の誤差が大きい原因として、検量線の縦軸(DNAの長さ)は対数をとった値を使用しており、DNAの長さが小さいほど、検量線から算出した値の定量的信頼性が低くなっていることが考えられる。また、ゲルに埃などのゴミが混入していたことも考えられる。

\section{考察}

\subsection{Ligation後の電気泳動}

\quad 課題5.6と同様に、Ligation後の電気泳動の結果から検量線を作成して、観測されたバンドから確認されたDNA断片の大きさを求めた。検量線を図\ref{fig:line_ligation}、各バンドの泳動距離を図\ref{tab:ligation}に、DNA断片を求めた結果を表\ref{tab:ligation_result}に示す。

\begin{figure}[htbp]
    \centering
    \includegraphics[width=8cm]{./data/line_ligation.png}
    \caption{Ligation後の電気泳動の検量線}
    \label{fig:line_ligation}
\end{figure}

\begin{table}[htbp]
    \centering
    \caption{Ligation後の電気泳動で確認されたバンドの泳動距離(px)}
    \begin{tabular}{|c|c|c|} \hline
        P, E, G & P, E & P, G \\ \hline
        196 & 199 & 151 \\
        300 &  & 200 \\ \hline
    \end{tabular}
    \label{tab:ligation}
\end{table}

\begin{table}[htbp]
    \centering
    \caption{Ligation後の電気泳動で確認されたバンドが示すDNAのサイズ(bp)}
    \begin{tabular}{|c|c|c|} \hline
        P, E, G & P, E & P, G \\ \hline
        3759 & 3604 & 7072 \\
        872 &  & 3554 \\ \hline
    \end{tabular}
    \label{tab:ligation_result}
\end{table}

また、pBlueScript, EGFP, GABA, pBlueScript + EGFP, pBlueScript + GABAのDNA断片の大きさの理論値を表\ref{tab:ligation_theory}に示す。

\begin{table}[htbp]
    \centering
    \caption{Ligation後の電気泳動で確認されたバンドが示すDNAのサイズ(bp)}
    \begin{tabular}{|c|c|c|c|c|} \hline
        pBlueScript & EGFP & GABA & pBlueScript + EGFP & pBlueScript + GABA \\ \hline
        2961 & 726 & 3648 & 3687 & 6647\\ \hline
    \end{tabular}
    \label{tab:ligation_theory}
\end{table}

以上の理論値と実測値を比較すると、pBlueScript + EGFP, pBlueScript + GABAに対応するDNAが確認できるため、Ligationが成功していると言える。

\subsection{PCR後の電気泳動}

\quad 本実験では、EGFPのPCR後の電気泳動で、バンドを確認することができなかった。また、考察6.1でLigation後のバンドの同定をした際に、pBlueScript + EGFPのバンドは確認できたことから、PCR自体には成功しているがバンドは観測されなかったことが言える。Ligation後の電気泳動でGABAに対応する大きさのバンドは観測されたが、EGFPは観測されなかったことから、PCRで使用した際にEGFPの希釈を過剰に行なってしまったことが原因と考えられる。


\begin{thebibliography}{99}
    \bibitem{1} ゲノムDNA精製システム選択ガイド, ThermoFisher, \url{https://www.thermofisher.com/blog/learning-at-the-bench/dna_nap_bid_ts_1}
    \bibitem{2} 制限酵素, 東邦大学, \url{https://www.toho-u.ac.jp/sci/biomol/glossary/bio/restriction_enzyme.html}
    \bibitem{3} クローニングの基礎知識, ThermoFisher, \url{https://www.thermofisher.com/jp/ja/home/life-science/cloning/cloning-learning-center/invitrogen-school-of-molecular-biology/molecular-cloning/cloning/traditional-cloning-basics.html}
    \bibitem{4} Loading buffer, ニッポン・ジーン株式会社, \url{https://www.nippongene.com/siyaku/product/buffer/electrophoresis/loading-buffer.html}
    \bibitem{5} ゲル電気泳動, 公益社団法人高分子学会, \url{https://www.spsj.or.jp/equipment/news/news_detail_36.html}
    \bibitem{6} Ethidium Bromide(EtBr), ベルトールドジャパン株式会社, \url{https://www.berthold-jp.com/ethidium-bromide/}
    \bibitem{7} What Is DNA Transformation, GenScript, \url{https://www.genscript.com/what-is-dna-transformation.html}
    \bibitem{8} 教育目的遺伝子組換え実験, 大藤道衛, \url{https://web.tuat.ac.jp/~idenshi/Japanese%20Files/Seminar_Folder/rikakyoin_Folder/H27_Folder/Archives%20H27/H27_kensyu345_1.pdf}
    \bibitem{9} 植物由来のIPTG, funakoshi, \url{https://www.funakoshi.co.jp/contents/51935}
    \bibitem{10} いまさら聞けないプラスミド抽出法の原理, 高木 昌宏, \url{https://www.sbj.or.jp/wp-content/uploads/file/sbj/8909/8909_yomoyama_1.pdf}
    \bibitem{11} エタノール沈殿あれこれ, 春木 満, \url{https://www.sbj.or.jp/wp-content/uploads/file/sbj/8905/8905_yomoyama-1.pdf}
    \bibitem{12} PCR, ニッポン・ジーン株式会社, \url{https://www.nippongene.com/siyaku/product/pcr/cat_pcr.pdf}
    \bibitem{13} 大腸菌の菌株の特徴を知ろう, 林 勇樹, \url{https://www.sbj.or.jp/wp-content/uploads/file/sbj/9401/9401_yomoyama.pdf}
    \bibitem{14} 制限酵素の主な注意点, ThermoFisher, \url{https://www.thermofisher.com/jp/ja/home/life-science/cloning/cloning-learning-center/invitrogen-school-of-molecular-biology/molecular-cloning/restriction-enzymes/restriction-enzyme-key-considerations.html}
\end{thebibliography}

\end{document} 