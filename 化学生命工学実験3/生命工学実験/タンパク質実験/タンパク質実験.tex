\documentclass{ltjsarticle}
\usepackage{amsmath}
\usepackage{amssymb}
\usepackage{graphicx} % Required for inserting images
\usepackage{enumerate}
\usepackage[version=4]{mhchem}
\usepackage{caption}
\usepackage{url}

\title{タンパク質実験}
\author{中村 優作}
\date{April 2023}

\pagestyle{plain}
\begin{document}
\flushleft{
    \huge{令和5年度 化学生命工学実験3}
    
    \vspace{100pt}
    
    \huge{タンパク質実験}
}

\vfill
\begin{flushright}

        \Large{\underline{学籍番号 : 082110424}}

        \vspace{30pt}
    
        \Large{\underline{氏名 : 中村優作}}

        \vspace{30pt}
    
        \Large{\underline{実施日 : 10月4日, 5日, 10日}}
        
\end{flushright}

\newpage

\section{目的}

\quad 本実験では、組換えタンパク質sfGFP-Hを発現させた大腸菌から、His-tagを用いたアフィニティクロマトグラフィにより、sfGFP-Hを生成する。得られたタンパク質を紫外可視分光法、ブラッドフォード法、SDS-PAGEにより定量する。これら3つの方法で算出された濃度を比較して考察することを目的とする。

% 操作-----------------------------------------------------------------------------
\section{操作}

\subsection{アフィニティクロマトグラフィによるsfGFP-Hの精製}

\quad 電気泳動用に、大腸菌破砕液$100\mathrm{\mu L}$を分注した。アフィニティクロマトグラフィ担体が詰められたカラム(Bio-Scale Mini Profinity IMACカートリッジ)に、表\ref{table:sfGFP-Hの精製手順}のように順次溶液を打ち込んだ。

\renewcommand{\tablename}{表}
\begin{table}[hbtp]
    \captionsetup[table]{skip=5pt}
    \centering
    \caption{sfGFP-Hの精製手順}
    \begin{tabular}{|l|l|c|c|}
        \hline
        ステップ & 使用buffer & 分取量($\mathrm{mL}$) & 使用チューブ($\mathrm{mL}$) \\
        \hline
        1. カラム平衡化 & IMAC B1 buffer & 8 & 15 \\
        2. タンパク質結合 & 大腸菌破砕液 & 2 & 15 \\
        3. カラム洗浄 & IMAC B2 buffer &  20 & 50 \\
        4. タンパク質溶出 & IMAC B3 buffer & 0.5 $\times$ 8本 & 15 \\
        \hline
    \end{tabular}
    \label{table:sfGFP-Hの精製手順}
\end{table}

\quad 打ち込む際は、プランジャを抜いたシリンジをカラムに装着し、シリンジ内に溶液をピペットで加え、プランジャを押し込むことで溶液をカラムに流し込んだ。また、カラムに溶液を流し込む速度は、1秒に2滴程度となるように調節した。また、打ち込む溶液を交換する際は、カラムからシリンジを外してからプランジャを抜いた。

\subsection{紫外・可視分光法によるタンパク質の濃度決定}

\subsubsection{測定に用いるタンパク質の調整}

\quad タンパク質溶出によって分取した8本のチューブのうち、目視で最も濃いと判断した2番目に溶出したチューブを溶出液として取り扱った。$1.5\mathrm{mL}$チューブに、溶出液$100\mathrm{\mu L}$を取り、$900\mathrm{\mu L}$のIMAC B3 bufferを加えることで、1/10倍に希釈した。

\subsubsection{分光光度計による吸光度の測定}

\quad 1/10倍に希釈した溶出液を、キュベットに移動し、分光光度計にかけて吸光度を測定した。測定波長は、$280\mathrm{nm}$と$488\mathrm{nm}$とした。$488\mathrm{nm}$で測定した際、吸光度が2を超え、定量的信頼性が下がると判断して、溶液を1/20倍希釈になるように調整した。1/20倍に希釈した溶液を、キュベットに移動し、分光光度計にかけて吸光度を測定した。

\subsection{ブラッドフォード法によるタンパク質の定量}

\quad タンパク質の標準サンプルとして、$8\mathrm{mg/mL}$のBSA溶液$20\mathrm{\mu L}$を用意した。このBSA溶液をIMAC B3 bufferを用いて系列希釈し、$0.8\mathrm{mg/mL}$, $0.4\mathrm{mg/mL}$, $0.2\mathrm{mg/mL}$, $0.1\mathrm{mg/mL}$, $0.05\mathrm{mg/mL}$の溶液各$50\mathrm{\mu L}$を$1.5\mathrm{mL}$チューブに調整した。

\quad また、溶出液$1\mathrm{\mu L}$に$49\mathrm{\mu L}$のIMAC B3 bufferを加えた、1/50x溶出液と、コントロール用の$50\mathrm{\mu L}$B3 bufferを$1.5\mathrm{mL}$チューブに調整した。

\subsubsection{ブラッドフォード溶液の混合}

\quad $1.5\mathrm{mL}$チューブ内で、$50\mathrm{\mu L}$のサンプルに$200\mathrm{\mu L}$のブラッドフォード試薬を加え、一旦混合した。その後、$750\mathrm{\mu l}$の超純水を添加した。この時、赤色から青色に溶液の色が変化した。

\subsubsection{分光光度計による吸光度の測定}

\quad 調整したブラッドフォード溶液の全量($1000\mathrm{\mu L}$)を用いて、吸光度の測定を$595\mathrm{nm}$で分光光度計を用いて行った。

\subsection{SDS ポリアクリルアミドゲル電気泳動によるsfGFP-Hの解析}

\subsubsection{泳動サンプルの調整}

\quad 表\ref{table:泳動サンプルの調整}のように、サンプルに1x SDS-PAGE用loading bufferを添加し、混合した後、$95\mathrm{^{\circ} C}$で2分間加熱して用意した。

\renewcommand{\tablename}{表}
\begin{table}[hbtp]
    \captionsetup[table]{skip=5pt}
    \centering
    \caption{泳動サンプルの調整}
    \begin{tabular}{|c|c|c|}
        \hline
        & サンプル & 1x SDS-PAGE用loading buffer \\
        \hline
        精製前 & $5\mathrm{\mu L}$ & $45\mathrm{\mu L}$ \\
        カラムの通過液 & $5\mathrm{\mu L}$ & $45\mathrm{\mu L}$ \\
        タンパク質溶出液(1/40) & $2\mathrm{\mu L}$ & $78\mathrm{\mu L}$ \\
        タンパク質溶出液(1/200) & $5\mathrm{\mu L}$タンパク質溶出液(1/40) & $20\mathrm{\mu L}$ \\
        \hline
    \end{tabular}
    \label{table:泳動サンプルの調整}
\end{table}

\subsubsection{SDS-PAGEの実施}

\quad ゲルを泳動槽にセットし、SDS-PAGE用泳動bufferを添加した。ゲルはATTO社のE-T 12.5L ePAGEL $12.5\%$を使用した。調整したサンプルをそれぞれ$4\mathrm{\mu L}$ずつ、ウェルに打ち込んだ。その後、ゲル1枚あたり$40\mathrm{mA}$の条件で40分程度通電した。

\quad ここでカラムの通過液をゲルに打ち忘れたか、精製前のサンプルのウェルに二重で打ち込んでしまったため、ゲルにバンドが3本のみしか見られていない。

\subsubsection{ゲルの染色と観察}

\quad ゲルを取り出し、染色用の容器に移した。あらかじめ温めておいた超純水を容器の半分程度に加えて、電子レンジで数秒間沸騰させた後、室温で5分静置した。超純水を交換し、同様の操作を計3回繰り返した。完全に水を切って、CBB染色液$10\mathrm{mL}$を加え、電子レンジで加熱し、沸騰するのが見えたら直ちに止め、室温で20分程度静置してバンドが染まったことを確認したら染色液を捨てた。超純水を半分程度加え、電子レンジで沸騰させた後、室温で5分静置した。超純水を交換し、同様の操作を計2回繰り返した。水を捨てた後、ゲルを撮影装置で撮影した。


% 結果--------------------------------------------------------------------------------------------------
\section{結果}

\subsection{紫外・可視分光法によるタンパク質の濃度決定}

\quad 1/20倍に希釈した溶液の吸光度を、$280\mathrm{nm}$と$488\mathrm{nm}$で測定した。吸光度の測定結果を表\ref{table:タンパク質溶出液の吸光度の測定結果}に示す。濃度の計算に関しては、設問/課題4.2に記載した。

\renewcommand{\tablename}{表}
\begin{table}[hbtp]
    \captionsetup[table]{skip=5pt}
    \centering
    \caption{タンパク質溶出液の吸光度の測定結果}
    \begin{tabular}{|c|c|}
        \hline
        測定波長 $\mathrm{nm}$ & 吸光度 \\
        \hline
        280 & 0.467 \\
        488 & 1.134 \\
        \hline
    \end{tabular}
    \label{table:タンパク質溶出液の吸光度の測定結果}
\end{table}

\subsection{ブラッドフォード溶液の、分光光度計による吸光度の測定}

\quad 調整したブラッドフォード溶液を用いて、吸光度の測定を$595\mathrm{nm}$で分光光度計を用いて行った結果を表\ref{table:ブラッドフォード溶液の吸光度の測定結果}に示す。また、測定結果から作成した検量線を図\ref{fig:吸光度とブラッドフォード溶液濃度の関係}に示す。なお、検量線の縦軸の吸光度はブランクを差し引いた値を用いている。

\renewcommand{\tablename}{表}
\begin{table}[hbtp]
    \captionsetup[table]{skip=5pt}
    \centering
    \caption{ブラッドフォード溶液の吸光度の測定結果}
    \begin{tabular}{|c|c|}
        \hline
        ブラッドフォード溶液 & 吸光度 \\
        \hline
        $0.8\mathrm{mg/mL}$ & 2.063 \\
        $0.4\mathrm{mg/mL}$ & 1.046 \\
        $0.2\mathrm{mg/mL}$ & 0.994 \\
        $0.1\mathrm{mg/mL}$ & 0.932 \\
        $0.05\mathrm{mg/mL}$ & 0.874 \\
        B3 buffer & 0.828 \\
        1/50x溶出液 & 1.151 \\
        \hline
    \end{tabular}
    \label{table:ブラッドフォード溶液の吸光度の測定結果}
\end{table}

\begin{figure}[hbtp]
    \centering
    \includegraphics[width=10cm]{ブラッドフォード検量線.png}
    \caption{吸光度とブラッドフォード溶液濃度の関係}
    \label{fig:吸光度とブラッドフォード溶液濃度の関係}
\end{figure}

作成した検量線から、1/50x溶出液のブラッドフォード溶液濃度を求めたところ、$0.2900\mathrm{mg/mL}$となった。これは50倍希釈の溶液なので、原液の重量濃度は$14.50\mathrm{mg/mL}$である。sfGFP-Hの分子量が$28180\mathrm{Da}$であることから、原液のモル濃度は以下のように求められた。

\begin{equation}
    c_{原液} = \frac{14.50\mathrm{mg/mL}}{28180\mathrm{Da}} = 51.4\mathrm{\mu M}
\end{equation}



\subsection{sfGFPの濃度決定} 

\quad 図\ref{fig:SDS-PAGEの結果}に、染色されたゲルの結果を示す。
\begin{figure}[hbtp]
    \centering
    \includegraphics[width=10cm]{result.png}
    \caption{SDS-PAGEの結果}
    \label{fig:SDS-PAGEの結果}
\end{figure}

\quad ImageJを用いて、標準用BSAのバンド4本と、1/40xタンパク質溶出液のバンド(図\ref{fig:SDS-PAGEの結果}の左から7番目)について、バンドの強度を定量した。その結果を表\ref{table:SDS-PAGEの結果}に示す。また、結果から作成した検量線を図\ref{fig:バンド強度とBSA濃度の関係}に示す。

\renewcommand{\tablename}{表}
\begin{table}[hbtp]
    \captionsetup[table]{skip=5pt}
    \centering
    \caption{SDS-PAGEの結果}
    \begin{tabular}{|c|c|c|}
        \hline
        バンド & 強度 & 濃度($\mathrm{mg/mL}$) \\
        \hline
        BSA(1) & 2.976 & 0.25\\
        BSA(2) & 2.457 & 0.125 \\
        BSA(3) & 1.659 & 0.0625 \\
        BSA(4) & 1.038 & 0.03125 \\
        1/40x溶出液 & 3.137 &  \\
        \hline
    \end{tabular}
    \label{table:SDS-PAGEの結果}
\end{table}

\begin{figure}[hbtp]
    \centering
    \includegraphics[width=10cm]{濃度検量線.png}
    \caption{バンド強度とBSA濃度の関係}
    \label{fig:バンド強度とBSA濃度の関係}
\end{figure}

作成した検量線から、1/40x溶出液のsfGFP濃度を求めたところ、$0.2494\mathrm{mg/mL}$であった。これは40倍希釈の溶液なので、原液の重量濃度は$9.976\mathrm{mg/mL}$である。sfGFPの分子量が$28180\mathrm{Da}$であることから、原液のモル濃度は以下のように求められた。

\begin{equation}
    c_{原液} = \frac{9.976\mathrm{mg/mL}}{28180\mathrm{Da}}= 35.4\mathrm{\mu M}
\end{equation}

\subsection{sfGFP-Hの分子量の決定}

\quad ImageJを用いて、BFPの泳動距離を測定し、各タンパク質のバンドの相対泳動度(Rf値)を測定した。各タンパク質の分子量とRf値を表\ref{table:各タンパク質の分子量とRf値}に示す。また、結果から作成した検量線を図\ref{fig:分子量とRf値の関係}に示す。

\renewcommand{\tablename}{表}
\begin{table}[hbtp]
    \captionsetup[table]{skip=5pt}
    \centering
    \caption{各タンパク質の分子量とRf値}
    \begin{tabular}{|c|c|c|}
        \hline
        分子量($\mathrm{kDa}$) & Rf値 \\
        \hline
        200 & 0.160361567 \\
        150 & 0.207231336 \\
        100 & 0.266488115 \\
        75 & 0.328757951 \\
        50 & 0.438901908 \\
        37 & 0.541680616 \\
        25 & 0.728824908 \\
        20 & 0.7941078 \\
        \hline
    \end{tabular}
    \label{table:各タンパク質の分子量とRf値}
\end{table}

\begin{figure}[hbtp]
    \centering
    \includegraphics[width=10cm]{分子量検量線.png}
    \caption{分子量とRf値の関係}
    \label{fig:分子量とRf値の関係}
\end{figure}

作成した検量線から、1/40xタンパク質溶出液、1/200xタンパク質溶出液の分子量はそれぞれ$24.8\mathrm{kDa}$, $25.5\mathrm{kDa}$であった。



% 課題---------------------------------------------------------------------------------

\section{設問/課題}

\subsection{sfGFPの$280\mathrm{nm}$におけるmol吸光係数$\epsilon_{280}$の計算}

\quad ExPasyのTranslateに与えられたsfGFPのアミノ酸配列を用いて、5'3'Frame1のアミノ酸配列を算出した。その後、ProtParamを用いて分子量と理論上のmol吸光係数を求めたところ、分子量は$2.8180 \mathrm{g/mol}$、理論上のmol吸光係数は$18910 \mathrm{M^{-1}cm^{-1}}$となった。

\subsection{吸光度を用いたsfGFP濃度の計算}

\quad 5.1で計算したmol吸光係数と、実測した$280\mathrm{nm}$における吸光度$A_{280}$の値を用いて、Lambert-Beerの式から、sfGFPの濃度を計算したところ以下のようになった。
\begin{equation}
    c_{20倍希釈} = \frac{0.467}{18910 \mathrm{M^{-1}cm^{-1}}} = 2.47 \mathrm{\mu M}
\end{equation}

これは20倍希釈したsfGFP溶液の濃度であるため、溶出液のsfGFP濃度は$49.4 \mathrm{\mu M}$となる。

\subsection{His-tag以外の精製用ペプチドtagとタンパク質tagの例}

\subsubsection{C-tag}

\quad C末端にのみ付加できるタグで、EPEAのアミノ酸配列を持つ。高濃度のSEPEAペプチドによる競合溶出法、または塩濃度が高く(例:2M MgCl2)かつ酸性条件の溶液で、タンパク質を変性させずに溶出する。\cite{1}

\subsubsection{GST-tag}

\quad 分子量は約26kDaで、211アミノ酸残機を持つタンパク質である。発現量が多く、細胞の制御機構に関与するタンパク質である。抗GST VHH抗体(Nanobody®)またはグルタチオンと強固に結合し、樹脂担体への残存は少ない傾向にある。還元型グルタチオンを用いて溶出する。GSTタグによってタンパク質の溶解性が向上し、可溶性タグ融合タンパク質の発現量が増加する。GSTタグは多量体からなるタンパク質複合体の精製には適していない。タグの分子量が大きいため、目的タンパク質の機能に干渉する可能性がある。したがって、精製ステップの後にタグを除去する場合がある。\cite{1}

\subsection{本実験操作以外のタンパク質濃度測定法}

% 3つの方法でタンパク質濃度を測定した。
% 紫外可視分光法、ブラッドフォード法、SDS-PAGEによる濃度決定である。

\quad 本実験では、紫外可視分光法、ブラッドフォード法、SDS-PAGEによる濃度決定の3つの方法を用いてタンパク質濃度を測定した。その他の方法として、蛍光法がある。蛍光法にはタンパク質中の第一級アミンとの結合により蛍光を発する試薬を用いて定量する方法と、タンパク質をコートする界面活性剤に結合して蛍光を発する方法の2つがある。蛍光法の利点として、高感度である、試料が少量で済む、大掛かりな装置や熟練した技術を必要としないことが挙げられる。\cite{2}

\quad 第一級アミンとの反応を利用したタンパク質の定量について述べる。Fluorescamineは元々傾向を生じないが、タンパク質中の第一級アミンと速やかに反応すると、青緑色の蛍光(極大蛍光波長:$495\mathrm{nm}$)を発する誘導体を形成する。この時の蛍光強度を測定し、標準タンパク質で作成した検量線と比較することによって、タンパク質の定量分析を行うことができる。タンパク質との反応に際して、過剰量の試薬は水との反応により速やかに蛍光を生じない産物に変換されるため、fluorescamineは溶液中のタンパク質濃度の測定に適している。一方で、トリスなどアミン系試薬により測定が妨害される場合がある。\cite{2}

\quad 定量範囲に関して、紫外可視分光法は$50$〜$2000\mathrm{\mu g/mL}$、ブラッドフォード法は$50$〜$2000\mathrm{\mu g/mL}$であるのに対し、Fluorescamineを用いた蛍光法では、$0.3$〜$13\mathrm{\mu g/mL}$であり、軽微量のタンパク質を高感度で検出する場合は蛍光法の方が適しているといえる。また、SDS-PAGEのようなゲル電気泳動法に比べて装置などが必要がなく簡便である。反応にアミン系試薬が必要な場合、測定が妨害される可能性があるため、蛍光法は使用するべきではない。\cite{2}

% 極微量のタンパク質を高感度で検出する場合、ブラッドフォードとかは検出感度が低いため、蛍光法で検出する方が望ましい。

\subsection{$488\mathrm{nm}$におけるmol吸光係数}

\quad 課題4.2から、溶出液中のsfGFP濃度は$24.7\mathrm{\mu M}$であった。また、結果3.1から、測定波長が$488\mathrm{nm}$での吸光度は$1.134$であった。Lambert-Beerの式から、$488\mathrm{nm}$でのmol吸光係数を求めると以下のようになる。

\begin{equation}
    \epsilon_{488} = \frac{1.134}{2.47 \mathrm{\mu M}} = 45911 \mathrm{M^{-1}cm^{-1}}
\end{equation}

\quad 文献(Creation, Expression, Purification and Characterization of GFP)によるとsfGFPの$\epsilon_{488}$の値は$83300\mathrm{M^{-1}cm^{-1}}$であった\cite{3}。実測値と比べると、値が大きく外れているといえる。

\quad 計算式は吸光度のみに依存しているので、吸光度の測定値が小さく観測されたために理論値と大きく外れたと考えられる。吸光度が小さい原因として、溶出液の希釈が過剰に行われてしまったことが考えられるが、希釈は$1.5\mathrm{mL}$チューブを用いて行っており、発生しうる誤差は$1\mathrm{\mu L}$程度であるため、希釈による誤差は考えにくい。

\subsection{SDS-PAGEの移動度から計算した分子量と理論上の分子量の不一致の原因}

\quad タンパク質の中には塩基性アミノ酸や酸性アミノ酸が極端に多いものもあり、SDSの結合量が標準的なタンパク質とは異なり、分子量から予想される移動度とは異なる挙動を示すタンパク質もある。また糖鎖などの翻訳後修飾によって見かけ上の移動度が影響されることもあり、SDS-PAGE による分子量の計算においては、このような誤差も考慮した解釈が必要となる。\cite{4}

% 考察--------------------------------------------------------------------------------
\section{考察}

\subsection{ブラッドフォード法での濃度決定}

\quad 結果3.2で示した、$0.8\mathrm{mg/mL}$ブラッドフォード溶液の吸光度は図\ref{fig:吸光度とブラッドフォード溶液濃度の関係}からもわかるように大きく外れた値をとっていると考えられる。吸光度は$log(1/T)$(Tは透過率)であるので、吸光度が$2.063$の場合、透過率は1%程度であり、定量的な信頼性が低いといえる。

\quad 外れ値を除いて溶出液の濃度を再度求める。$0.8\mathrm{mg/mL}$での吸光度の値を除いて検量線を作成したところ図\ref{fig:外れ値}のようになった。

\begin{figure}[hbtp]
    \centering
    \includegraphics[width=10cm]{外れ値.png}
    \caption{吸光度とブラッドフォード溶液濃度の関係(外れ値を除く)}
    \label{fig:外れ値}
\end{figure}

作成した検量線から、1/50x溶出液のブラッドフォード溶液濃度を求めたところ、$0.6005\mathrm{mg/mL}$となった。これは50倍希釈の溶液なので、原液の重量濃度は$30.025\mathrm{mg/mL}$である。sfGFPの分子量が$28180\mathrm{Da}$であることから、原液のモル濃度は以下のように求められた。

\begin{equation}
    c_{原液} = \frac{30.025\mathrm{mg/mL}}{28180\mathrm{Da}} = 106.5\mathrm{\mu M}
\end{equation}

外れ値を除いていない場合に比べて、2倍近く濃度が高い結果が得られた。吸光光度法でのモル濃度が$49.4\mathrm{\mu M}$、SDS-PAGEで$35.4\mathrm{\mu M}$であったことと比較すると、外れ値を除かない場合のモル濃度($51.4\mathrm{\mu M}$)の方がより正確な濃度に近いと考えられる。

\quad より正確に測定するため、最大の吸光度が1.5を超えない希釈率で測定をするべきであった。

\subsection{sfGFP-Hの分子量}

\quad 結果3.4で算出した、1/40xタンパク質溶出液、1/200xタンパク質溶出液の分子量はそれぞれ$24.8\mathrm{kDa}$, $25.5\mathrm{kDa}$であった。これはsfGFPの分子量($28180\mathrm{Da}$)と比較すると、大きく外れているといえる。これは、SDS-PAGEによる定量では、タンパク質の移動度から分子量を算出し、それを標準タンパク質の移動度と比較することで濃度を算出しているため、移動度の測定誤差が濃度の算出に影響していると考えられる。

\quad 結果3.3で算出した、1/40x溶出液のsfGFP濃度は$35.4\mathrm{\mu M}$であった。結果3.4で算出された1/40xタンパク質溶出液の分子量が$24.8\mathrm{kDa}$であることから、原液のモル濃度を再度計算すると以下のようになった。

\begin{equation}
    c_{原液} = \frac{9.976\mathrm{mg/mL}}{24.8\mathrm{kDa}} = 40.2\mathrm{\mu M}
\end{equation}


\subsection{それぞれの濃度測定の結果}

\quad 本実験では、紫外可視分光法、ブラッドフォード法、SDS-PAGEによる3つの方法でタンパク質濃度を測定した。それぞれの方法で測定した濃度を表\ref{table:それぞれの濃度測定の結果}に示す。

\renewcommand{\tablename}{表}
\begin{table}[hbtp]
    \captionsetup[table]{skip=5pt}
    \centering
    \caption{それぞれの濃度測定の結果}
    \begin{tabular}{|c|c|}
        \hline
        測定方法 & 濃度($\mathrm{\mu M}$) \\
        \hline
        紫外可視分光法 & 49.4 \\
        ブラッドフォード法 & 51.4 \\
        SDS-PAGE & 35.4 (40.8) \\
        \hline
    \end{tabular}
    \label{table:それぞれの濃度測定の結果}
\end{table}

\quad 結果から、紫外可視分光法とブラッドフォード法により算出された濃度は比較的近い値を取ったが、SDS-PAGEによる定量では他の2つの方法と比べて濃度が低く算出された。これは、SDS-PAGEによる定量では、タンパク質の移動度から分子量を算出し、それを標準タンパク質の移動度と比較することで濃度を算出しているため、移動度の測定誤差が濃度の算出に影響していると考えられる。また、SDS-PAGEで算出した分子量では、sfGFPの分子量($28180\mathrm{Da}$)と比較すると大きく外れていることから、移動度の測定誤差が大きいと考えられる。

\quad しかし、紫外可視分光法では課題4.5で述べたようにmol吸光係数が大きくずれてしまったこと、ブラッドフォード法では吸光度が外れた値をとっており検量線の信頼度が低いと考えられるため、どの方法が最も正確な濃度を測定できたかは明確ではない。


\begin{thebibliography}{99}
    \bibitem{1} タンパク質精製用タグ, 株式会社プロテインテックジャパン, \url{https://www.ptglab.co.jp/news/blog/tags-for-protein-purification/}
    \bibitem{2} 総タンパク質の定量法, 鈴木 祥夫, \url{https://www.jsac.or.jp/bunseki/pdf/bunseki2018/201801nyuumon.pdf}
    \bibitem{3} Creation, Expression, Purification and Characterization of GFP \url{https://www.biophysik.physik.uni-muenchen.de/teaching/laboratory_courses/gfp_expression/g4b_gfp_expressionenglish_2017.pdf}
    \bibitem{4} SDS-PAGEのキホン!分子量計算, バイオ・ラッドTechnical QA, \url{https://pdbu-support.bio-rad.co.jp/techbrief/bulletins201804.html}
\end{thebibliography}




\end{document} 