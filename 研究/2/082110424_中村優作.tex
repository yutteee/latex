\documentclass{ltjsarticle}
\usepackage{amsmath}
\usepackage{amssymb}
\usepackage{graphicx} % Required for inserting images
\usepackage{enumerate}
\usepackage[version=4]{mhchem}
\usepackage{caption}
\usepackage{url}

\title{実験レポテンプレ}
\author{中村 優作}
\date{April 2023}

\pagestyle{plain}
\begin{document}

\section{結果}

\quad 表\ref{table:1}にDCA濃度と各プラスミド添加量条件を、図\ref{fig:1}にホタルルシフェラーゼアッセイの結果、図\ref{fig:2}に$\beta$-galアッセイの結果、図\ref{fig:3}にLuc activityの結果を示す。


\renewcommand{\tablename}{表}
\begin{table}[hbtp]
    \captionsetup[table]{skip=5pt}
    \centering
    \caption{DCA濃度と各プラスミド添加量条件}
    \begin{tabular}{|c|c|c|}
        \hline
        Sample & DCA concentration (\textmu M) & Amount of each plasmids (ng) \\
        \hline
        1 & 0 & 0 \\
        2 & 0 & 500 \\
        3 & 20 & 0 \\
        4 & 20 & 500 \\
        \hline
    \end{tabular}
    \label{table:1}
\end{table}

\begin{figure}[hbtp]
    \centering
    \includegraphics[width=0.7\textwidth]{./luc.png}
    \caption{ホタルルシフェラーゼアッセイの結果}
    \label{fig:1}
\end{figure}

\begin{figure}[hbtp]
    \centering
    \includegraphics[width=0.7\textwidth]{./gal.png}
    \caption{$\beta$-galアッセイの結果}
    \label{fig:2}
\end{figure}

\begin{figure}[hbtp]
    \centering
    \includegraphics[width=0.7\textwidth]{./relative_luc.png}
    \caption{Luc activityの結果}
    \label{fig:3}
\end{figure}

\newpage

\section{考察}

\subsection{トランスフェクション時の細胞の剥がれ}

\quad Opti-MEM、DNA-Lipofectamine複合液の添加をした際、一部の細胞がwellから剥がれてしまった。これには以下の原因が考えられる。

\begin{enumerate}[(1)]
    \item 継代が不十分\\
    \quad トランスフェクション前の継代で、$3.0 \times 10^5 \mathrm{cells/well}$で1日培養をしたが、細胞数はwellの80\%程度であった。このため、細胞がwellから剥がれやすくなっており、溶媒を添加したことで簡単に剥がれてしまったと考えられる。1日で十分に細胞を増やすために、$3.5 \times 10^5 \mathrm{cells/well}$程度の細胞数で継代を行うとよいと考えられる。
    \item 不適切な添加操作\\
    \quad 顕微鏡で溶媒添加後のwellの状態を確認したところ、全てのwellでwellの右側が剥がれていた。溶媒滴下時には、wellを左に傾けた状態で左側にゆっくりと溶媒を滴下した。この操作で右側の細胞が剥がれた原因として、傾けたwellを戻す時にwellの右側に溶媒が流れ込んで剥がれた可能性がある。傾けたwellを戻す際に、ゆっくりと戻す様に注意する必要がある。
\end{enumerate}

\subsection{Luc activity}

\quad 図\ref{fig:1}から、プラスミドが入っていない系では、ホタルルシフェラーゼアッセイの活性が見られなかったため、遺伝子の発現が起こっていないことが確認された。図\ref{fig:3}より、プラスミドが入っている系では、DCAが含まれているとLuc activityが大幅に大きくなっているため、DCAがLuc activityの増加に寄与していることが言える。これは、DCAがhTGR5に結合することで細胞内cAMPが増加し、CREと結合してCREBが活性化し、ホタルルシフェラーゼ遺伝子の転写が誘導されるためと考えられる。

\end{document} 