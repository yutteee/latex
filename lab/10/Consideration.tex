\documentclass{ltjsarticle}
\usepackage{amsmath}
\usepackage{amssymb}
\usepackage{graphicx} % Required for inserting images
\usepackage{enumerate}
\usepackage[version=4]{mhchem}
\usepackage{caption}
\usepackage{url}

\title{実験レポテンプレ}
\author{中村 優作}
\date{April 2023}

\pagestyle{plain}
\begin{document}


\section{結果}

WYT, CGEの2配列とも沈殿を得ることができなかった。

\section{考察}

% - 石井さんと一緒にやって、石井さんの6残基程度のペプチドは沈殿を得ることができていた
% - また、-80度で放置したら沈殿が出た
% - そのため、ペプチド合成までの操作は問題ない
% - 問題があるとすればペプチドの脱保護、沈殿の操作
% - 攪拌を十分にし、水の量を1mLにしたが、沈殿を得ることができなかった。
% - 再度横野さんに問い合わせた。
% - Tripeptideでは、チオアニソール、フェノール、EDTは不要であり、これが、Heptan/TBMEと相互作用をもつ可能性があります。92.5% TFA, 2.5% Water, 5.0% TISのカクテルをお勧めいたします。
% Tripeptideの沈殿はシーケンスに大きく依存し、沈殿しにくいものは、Heptan/TBMEの量を増やす必要があります。この液を現在の倍量にしてお試しいただければと思います。
% 水の量は、750uLに対して2-3滴(5-10uL)ですので、heptan/TBMEが分離しない範囲で入れていただければと思います。分離した場合は、沈殿が困難になります。
% - 水が今回は分離していたため、次回は分離しない範囲で水を加える
% - また、Heptan/TBMEの量を増やすこと, カクテルを変更することを検討する。

石井さんと共同で実験を行い、石井さんが使用した6残基程度のペプチドにおいて沈殿を得ることができた。また、-80度での保存によっても沈殿が確認された。この結果から、ペプチド合成に関する操作には問題がないと考えられる。

しかし、ペプチドの脱保護および沈殿に関する操作には改善の余地があることが言える。この問題について再度横野さんに相談した結果、次の提案をいただいた。Tripeptideに関しては、チオアニソール、フェノール、EDTの使用は不要であり、これらがHeptan/TBMEと相互作用を持つ可能性が指摘された。92.5\% TFA、2.5\% Water、5.0\% TISのカクテルを使用することが推奨された。また、沈殿のしやすさはシーケンスに依存するため、沈殿が得られにくい場合にはHeptan/TBMEの量を増やす必要がある。特に、現在使用している量の倍量を試してみることが推奨された。さらに、水の量は750µLに対して2-3滴(5-10µL)であり、Heptan/TBMEが分離しない範囲で水を加えることが推奨された。分離が起こると、沈殿が困難になる可能性があるためである。

今回の実験では、水が分離してしまったため、次回は分離しない範囲で水を加えることを検討する。また、カクテルの組成を変更すること、Heptan/TBMEの量を増やすことも検討する。

\end{document} 