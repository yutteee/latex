\documentclass{ltjsarticle}
\usepackage{amsmath}
\usepackage{amssymb}
\usepackage{graphicx} % Required for inserting images
\usepackage{enumerate}
\usepackage[version=4]{mhchem}
\usepackage{caption}
\usepackage{url}
\usepackage{comment}


\title{実験レポテンプレ}
\author{中村 優作}
\date{April 2023}

\pagestyle{plain}
\begin{document}


\section{結果}

図\ref*{fig:result}に、各ペプチドのTGR5活性評価の結果を示す。いくつかのペプチドは,TGR5に対する小さい活性を示していることがわかる。最も強い活性を示したのはPPCで、DCAのおよそ25\%程度の活性を示した。

\begin{figure}[h]
    \centering
    \includegraphics[width=1\textwidth]{result.png}
    \caption{各ペプチドのTGR5活性評価の結果}
    \label{fig:result}
\end{figure}

\section{考察}

\begin{comment}
- 上位が全部そんなに示さなかった
- 1位が全然ダメ
- モデルはPPという構造をTGR5活性に対して重要視しているが、この結果から重要でないことがわかる
- 活性が出てるものと出ていないものの違いを考えてみる
- 生物物理学的特性を見てみること
- 過去のペプチドと比較すること
- the isoelectric point in neutral pH, (2) GRAVY (grand average of hydropathy)46, (3) the instability index47, (4) aromaticity, (5) the molar extinction coefficient and (6) molecular weight.
- 今回DCAの誤差が大きいので単純な比較をしていいのか微妙だが、清水さんのDCA結合ペプチドの結果では30%程度のものはあった。
- PPが重要な構造であるという仮説がおそらく間違っており、なんでPPが重要な構造とモデルは示したのかを考える必要がある。
- 重要な寄与を示すパラメータを抽出して、それらが上位に含まれるかどうかを確認してみること。
\end{comment}



\end{document} 