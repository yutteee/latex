\documentclass{ltjsarticle}
\usepackage{amsmath}
\usepackage{amssymb}
\usepackage{graphicx} % Required for inserting images
\usepackage{enumerate}
\usepackage[version=4]{mhchem}
\usepackage{caption}
\usepackage{url}

\title{実験レポテンプレ}
\author{中村 優作}
\date{April 2023}

\pagestyle{plain}
\begin{document}

\begin{flushleft}
    \huge{3残基ペプチドの合成}
\end{flushleft}

\section{自動合成}

$15.603\mathrm{µmol}$スケールで、AWF, IPFの2配列を自動合成する。


\section{脱保護}

\begin{enumerate}[1.]
    \item シリンジに、カラムを反対向きで口に当て、シールごと樹脂をシリンジに移す。
    \begin{description}
        \item[補足] 続けて操作しない場合は、$-20\mathrm{^{\circ}C}$で保存する。 
    \end{description}
    \item 脱保護試薬を表\ref*{table:1}の組成で作成する。
    \item シリンジにテフロンニードルを設置
    \item 遠沈管からカクテルを脱保護試薬を2mL程度まで吸う。
    \item テフロンニードルの先を遠沈管に差したまま、シリンジから空気を抜く。
    \item 3時間ドラフトシェーカーに放置する。
\end{enumerate}

\renewcommand{\tablename}{表}
\begin{table}[hbtp]
    \captionsetup[table]{skip=5pt}
    \centering
    \caption{脱保護試薬の組成}
    \begin{tabular}{|c|c|c|c|}
        \hline
        試薬 & 割合 (\%) & 実際量 (mL) \\
        \hline
        TFA & 81.5 & 24.45 \\
        水 & 10 & 3 \\
        TIS & 1 & 0.3 \\
        チオアニソール & 5 & 1.5 \\
        EDT & 2.5 & 0.75 \\
        \hline
    \end{tabular}
    \label{table:1}
\end{table}

\section{ペプチド沈殿}

\begin{enumerate}[1.]
    \item 脱保護試薬が入っていた$50\mathrm{mL}$遠沈管に、シリンジからペプチド溶液を出す。
    \item 遠沈管に15\%ヘプタン-TBME溶液で$30\mathrm{mL}$までメスアップする。
    \item $-20\mathrm{^{\circ}C}$で一晩放置する。
    \item 沈殿が得られなかったら、少量の水を添加し、ボルテックスして$4\mathrm{^{\circ}C}$で一晩放置する。
    \item 沈殿が得られたら、$0\mathrm{^{\circ}C}$, $3000\mathrm{g}$で10分間遠心する。
    \item 上清を捨て、蓋を開けて放置して乾燥させる。
\end{enumerate}

\section{凍結乾燥}

\end{document}