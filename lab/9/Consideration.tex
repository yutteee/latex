\documentclass{ltjsarticle}
\usepackage{amsmath}
\usepackage{amssymb}
\usepackage{graphicx} % Required for inserting images
\usepackage{enumerate}
\usepackage[version=4]{mhchem}
\usepackage{caption}
\usepackage{url}

\title{実験レポテンプレ}
\author{中村 優作}
\date{April 2023}

\pagestyle{plain}
\begin{document}


\section{結果}

AWF, IPFの2配列とも沈殿を得ることができなかった。

\section{考察}

% - 石井さんと一緒にやって、石井さんの6残基程度のペプチドは沈殿を得ることができていた
% - そのため、ペプチド合成までの操作は問題ない
% - 問題があるとすればペプチドの脱保護、沈殿の操作
% - 横野さんに問い合わせたところ、の15\%ヘプタン-TBMEを加えたあとに、充分に攪拌されましたでしょうか?ボルテックスまたは上下反転で、充分に攪拌する必要があります。2の水(1mLで十分かと思います)を加えたあとも同様です。という返信
% - 攪拌は実際に十分に行っていなかった
% - また、水の量が多かった可能性
% - また、沈殿しにくいペプチドだった可能性
% - そのため、次回はペプチドを変更すること、攪拌を十分に行うこと、水の量を変えることを検討する。

今回の実験では、石井さんと共同で行い、石井さんの6残基程度のペプチドについては沈殿を得ることができた。このことから、ペプチド合成までの操作に問題はないと考えられる。問題があるとすれば、ペプチドの脱保護や沈殿操作の段階である可能性が示唆された。

横野さんに問い合わせた結果、15\%ヘプタン-TBMEを加えた後に、十分に攪拌が行われているかが重要であり、ボルテックスや上下反転での攪拌が必要とのアドバイスをいただいた。また、2mLの水を加える際にも同様に十分な攪拌が必要であるとのことだったが、今回の実験では、攪拌が実際には十分に行われていなかった。また、水の量が多すぎた可能性や、使用したペプチドが沈殿しにくい性質を持っていた可能性も考えられる。

これらの要因を踏まえ、次回の実験ではペプチドを変更すること、攪拌を十分に行うこと、水の量を適切に調整することを検討する。

\end{document} 