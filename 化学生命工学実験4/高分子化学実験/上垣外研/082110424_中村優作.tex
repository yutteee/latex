\documentclass{ltjsarticle}
\usepackage{amsmath}
\usepackage{amssymb}
\usepackage{graphicx} % Required for inserting images
\usepackage{enumerate}
\usepackage[version=4]{mhchem}
\usepackage{caption}
\usepackage{url}

\title{実験レポテンプレ}
\author{中村 優作}
\date{April 2023}

\pagestyle{plain}
\begin{document}
\flushleft{
    \huge{令和5年度 化学生命工学実験4}
    
    \vspace{100pt}
    
    \huge{ラジカル共重合、リビングラジカル重合、界面重縮合}
}

\vfill
\begin{flushright}

        \Large{\underline{学籍番号 : 082110424}}

        \vspace{20pt}
    
        \Large{\underline{氏名 : 中村優作}}

        \vspace{20pt}

        \Large{\underline{班 : B-3}}

        \vspace{20pt}
    
        \Large{\underline{実施日 : 2024/1/11/, 1/16, 1/17}}
        
\end{flushright}

\newpage

\section{目的}
\section{操作}

\subsection{MMAとStのラジカル共重合}

\quad 試験管に、メタクリル酸メチル(MMA)$3.14\mathrm{mL} (2.94\mathrm{mmol})$、スチレン(St)$1.45\mathrm{mL} (12.6\mathrm{mmol})$、アゾビスイソブチロニトリル(AIBN)水溶液$1.34\mathrm{mL}$ (AIBN:$0.06\mathrm{mmol}$)を加えた。セプタムで蓋をして密閉した。試験管を氷浴に入れて\ce{N2}で3回排気、充填をした。その後、$120\mathrm{^{\circ}C}$のオイルバスに入れ30分間攪拌をした。その後、氷浴に入れて重合をクエンチした。

\quad $500\mathrm{mL}$ビーカーにメタノール$200\mathrm{mL}$を加え、マグネティックスターラーで攪拌をした。ピペットを用いて、攪拌中のメタノールにゆっくりと滴下したところ、白い沈殿が析出した。吸引濾過で生成物を回収した。生成物を$6\mathrm{mL}$のクロロホルムに溶かした。$300\mathrm{mL}$ビーカーに$200\mathrm{mL}$のメタノールを用意し、マグネティックスターラーで攪拌をした。そこに調整したクロロホルム溶液をゆっくり加えたところ、白い沈殿が析出した。吸引濾過で生成物を回収し、濾紙で乾燥させた。生成物の重量を測定したところ、$0.7986\mathrm{g}$であった。

\subsection{MMAとStのリビングラジカル重合}

\quad 試験管に、メタクリル酸メチル(MMA)$3.14\mathrm{mL} (2.94\mathrm{mmol})$、スチレン(St)$1.45\mathrm{mL} (12.6\mathrm{mmol})$、St-TEMPO水溶液$1.34\mathrm{mL}$ (AIBN:$0.06\mathrm{mmol}$)を加えた。セプタムで蓋をして密閉した。試験管を氷浴に入れて\ce{N2}で3回排気、充填をした。その後、$120\mathrm{^{\circ}C}$のオイルバスに入れ24時間攪拌をした。その後オイルバスから取り出し、氷浴に入れて重合を急冷した。

\quad $500\mathrm{mL}$ビーカーにメタノール$200\mathrm{mL}$を加え、マグネティックスターラーで攪拌をした。反応溶液に\ce{CHCl3}を$4\mathrm{mL}$加えた。この反応溶液を撹拌中のメタノールに対してゆっくり滴下した。吸引濾過でポリマーを回収したところ、??であった。

\subsection{界面重縮合}

\quad \ce{Na2CO3-10H2O}を($2.5\mathrm{g}$)とヘキサメチレンジアミン$2.0\mathrm{g}$($17\mathrm{mmol}$)を$200\mathrm{mL}$ビーカーに加えた。このビーカーに蒸留水$50\mathrm{mL}$を加え溶かした。また、$300\mathrm{mL}$ビーカーに、セバコイルクロリド$5\mathrm{mL}$($23\mathrm{mmol}$)、四塩化炭素$50\mathrm{mL}$を加え溶液を調整した。

\quad ヘキサメチレンジアミンの水溶液をセバコイルクロリドの溶液にゆっくりと加えた。界面で反応が起こり白い膜(ナイロン)が生成した。ガラス棒でナイロンを巻き取り、全て回収した。得られたポリマーを水で洗浄し、1日乾燥させた。

\section{結果}
\section{考察}
\section{設問/課題}
\begin{thebibliography}{99}
    % \bibitem{1} hogehoge \url{https://www.google.com/}
\end{thebibliography}

\end{document} 