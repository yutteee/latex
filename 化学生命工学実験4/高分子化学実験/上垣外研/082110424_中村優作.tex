\documentclass{ltjsarticle}
\usepackage{amsmath}
\usepackage{amssymb}
\usepackage{graphicx} % Required for inserting images
\usepackage{enumerate}
\usepackage[version=4]{mhchem}
\usepackage{caption}
\usepackage{url}

\title{実験レポテンプレ}
\author{中村 優作}
\date{April 2023}

\pagestyle{plain}
\begin{document}
\flushleft{
    \huge{令和5年度 化学生命工学実験4}
    
    \vspace{100pt}
    
    \huge{ラジカル共重合、リビングラジカル重合、界面重縮合}
}

\vfill
\begin{flushright}

        \Large{\underline{学籍番号 : 082110424}}

        \vspace{20pt}
    
        \Large{\underline{氏名 : 中村優作}}

        \vspace{20pt}

        \Large{\underline{班 : B-3}}

        \vspace{20pt}
    
        \Large{\underline{実施日 : 2024/1/11/, 1/16, 1/17}}
        
\end{flushright}

\newpage

\section{目的}

\quad 高分子は、プラスチックや繊維など日常で幅広く利用される材料を構成している。高分子を合成する方法には様々なものがあるが、本実験ではラジカル共重合、リビングラジカル重合、界面重縮合の3つの方法を用いて高分子を合成する。それぞれの方法で得られたポリマーを測定し、考察することを目的とする。
\section{操作}

\subsection{MMAとStのラジカル共重合}

\quad 試験管に、メタクリル酸メチル(MMA)$3.14\mathrm{mL} (2.94\mathrm{mmol})$、スチレン(St)$1.45\mathrm{mL} (12.6\mathrm{mmol})$、アゾビスイソブチロニトリル(AIBN)水溶液$1.34\mathrm{mL}$ (AIBN:$0.06\mathrm{mmol}$)を加えた。セプタムで蓋をして密閉した。試験管を氷浴に入れて\ce{N2}で3回排気、充填をした。その後、$120\mathrm{^{\circ}C}$のオイルバスに入れ30分間攪拌をした。その後、氷浴に入れて重合をクエンチした。

\quad $500\mathrm{mL}$ビーカーにメタノール$200\mathrm{mL}$を加え、マグネティックスターラーで攪拌をした。ピペットを用いて、攪拌中のメタノールにゆっくりと滴下したところ、白い沈殿が析出した。吸引濾過で生成物を回収した。生成物を$6\mathrm{mL}$のクロロホルムに溶かした。$300\mathrm{mL}$ビーカーに$200\mathrm{mL}$のメタノールを用意し、マグネティックスターラーで攪拌をした。そこに調整したクロロホルム溶液をゆっくり加えたところ、白い沈殿が析出した。吸引濾過で生成物を回収し、濾紙で乾燥させた。生成物の重量を測定したところ、$0.799\mathrm{g}$であった。

\subsection{MMAとStのリビングラジカル重合}

\quad 試験管に、メタクリル酸メチル(MMA)$3.14\mathrm{mL} (2.94\mathrm{mmol})$、スチレン(St)$1.45\mathrm{mL} (12.6\mathrm{mmol})$、St-TEMPO水溶液$1.34\mathrm{mL}$ (AIBN:$0.06\mathrm{mmol}$)を加えた。セプタムで蓋をして密閉した。試験管を氷浴に入れて\ce{N2}で3回排気、充填をした。その後、$120\mathrm{^{\circ}C}$のオイルバスに入れ24時間攪拌をした。その後オイルバスから取り出し、氷浴に入れて重合を急冷した。

\quad $500\mathrm{mL}$ビーカーにメタノール$200\mathrm{mL}$を加え、マグネティックスターラーで攪拌をした。反応溶液に\ce{CHCl3}を$4\mathrm{mL}$加えた。この反応溶液を撹拌中のメタノールに対してゆっくり滴下した。吸引濾過でポリマーを回収したところ、$1.54\mathrm{g}$であった。

\subsection{界面重縮合}

\quad \ce{Na2CO3-10H2O}を($2.5\mathrm{g}$)とヘキサメチレンジアミン$2.0\mathrm{g}$($17\mathrm{mmol}$)を$200\mathrm{mL}$ビーカーに加えた。このビーカーに蒸留水$50\mathrm{mL}$を加え溶かした。また、$300\mathrm{mL}$ビーカーに、セバコイルクロリド$5\mathrm{mL}$($23\mathrm{mmol}$)、四塩化炭素$50\mathrm{mL}$を加え溶液を調整した。

\quad ヘキサメチレンジアミンの水溶液をセバコイルクロリドの溶液にゆっくりと加えた。界面で反応が起こり白い膜(ナイロン)が生成した。ガラス棒でナイロンを巻き取り、全て回収した。得られたポリマーを水で洗浄し、1日乾燥させた。その後重量を測定したところ、$15.8\mathrm{g}$であった。

\section{結果及び考察}

\subsection{MMAとStのラジカル共重合}

\subsubsection{生成物のSECによる測定結果}

\quad 図\ref{fig:0}に、生成物のSECによる測定結果を示す。数平均分子量$M_n=35593$, 分散度$\frac{M_w}{M_n}=1.68$であった。

\begin{figure}[htbp]
    \centering
    \includegraphics[width=0.9\textwidth]{./data/1/sec.png}
    \caption{MMAとStのラジカル共重合のSECによる測定結果}
    \label{fig:0}
\end{figure}

\subsubsection{課題1 収率}

\quad 本実験では、MMA$2.94\mathrm{mmol}$(分子量$100\mathrm{g/mol}$), スチレン$12.6\mathrm{mmol}$(分子量$104\mathrm{g/mol}$)を用いた。これらが全てポリマーとして導入された場合に得られるポリマーの質量は、
\begin{align*}
    2.94\mathrm{mmol}\times100\mathrm{g/mol} + 12.6\mathrm{mmol}\times104\mathrm{g/mol} = 1.60\mathrm{g}
\end{align*}

である。また、実際に得られたポリマーの質量は$0.799\mathrm{g}$であった。したがって、収率は以下のようになる。
\begin{align*}
    \frac{0.799\mathrm{g}}{1.60\mathrm{g}}\times100\% = 49.9\%  
\end{align*}

\subsubsection{課題2 $^1$H NMRスペクトルの帰属}

\quad 図\ref{fig:1}に、$^1$H NMRスペクトルの結果ととそのピークの帰属を示す。

\begin{figure}[htbp]
    \centering
    \includegraphics[width=0.9\textwidth]{./data/1/nmr.png}
    \caption{MMAとStのラジカル共重合の$^1$H NMRスペクトル}
    \label{fig:1}
\end{figure}

\subsubsection{課題3 生成物の組成の計算}

\quad $7\mathrm{ppm}$付近のピークはスチレンのベンゼン環上のピークであることと、その他のピークは重なりが多く正確には帰属することができないことから、MMA(x)とスチレン(y)由来の積分比に対して以下の式が成り立つ。

\begin{align*}
    \left\{
    \begin{array}{l}
        5y = 5.66 - 0.66  \\
        8x + 3y = 3.42 + 1.73 + 0.98 +  5.02 + 1.00 + 5.31 - 0.30 \\
    \end{array}
    \right.
\end{align*}

これを解くと、$x = 1.77$, $y = 1$となる。したがって、生成物の組成は以下のようになる。
\begin{align*}
    MMA : St = 63.9 : 36.1
\end{align*}

\subsubsection{課題4 モノマーの反応性比}

\quad 異なる原料比を用いた他班の実験結果を用いて、モノマーの反応性比を求める。他班を含めた生成物の組成の実験結果を表\ref{tab:1}に示す。

\begin{table}[htbp]
    \centering
    \caption{他班を含めた生成物の組成の実験結果}
    \label{tab:1}
    \begin{tabular}{|c|c|} \hline
        仕込み比 (MMA / St) & 生成物の組成 (MMA / St) \\ \hline
        80/20 &  71/29 \\ \hline
        70/30 &  64/36 \\ \hline
        50/50 &  49/51 \\ \hline
        30/70 &  32/68 \\ \hline
        20/80 &  23/77 \\ \hline
    \end{tabular}
\end{table}

この結果と、式(\ref{eq:1})に示すMayo-Lewisの式を用いて、モノマーの反応性比を求める。

\begin{align}
    \frac{d[M_1]}{d[M_2]} = \frac{[M_1]}{[M_2]}(\frac{r_1[M_1] + [M_2]}{[M_1] + r_2[M_2]})    \label{eq:1}
\end{align}

ここで、$M_1, M_2$はそれぞれMMA, Stであり、$r_1, r_2$はそれぞれMMA, Stの反応性比である。$F = \frac{[M_1]}{[M_2]}$, $f = \frac{d[M_1]}{d[M_2]}$と仮定すると、式(\ref{eq:1})は以下のように表せる。

\begin{align}
    \frac{F(f-1)}{f} = r_1\frac{F^2}{f} - r_2  \label{eq:2}
\end{align}

式(\ref{eq:2})と、プロットした近似曲線を比較することで反応性比を求めることができる。図\ref{fig:2}に、プロットした近似曲線を示す。


\begin{figure}[htbp]
    \centering
    \includegraphics[width=0.6\textwidth]{./data/1/plot.png}
    \caption{$\frac{F^2}{f}$と$\frac{F(f-1)}{f}$の関係}
    \label{fig:2}
\end{figure}

図\ref{fig:2}より、反応性比は$r_1 = 0.47$, $r_2 = 0.60$となる。

\subsubsection{課題5 共重合組成曲線}

\quad ポリマー中へのモノマー$M_1$の導入率を$y$と仮定すると、$M_2$の導入率は$1-x$と表される。また、モノマー仕込み比におけるモノマー$M_1$の割合を$x$と仮定すると、モノマー$M_2$の割合は$1-x$と表される。これと、式(\ref{eq:1})を用いると、以下の式が成り立つ。

\begin{align}
    y = \frac{x(1-(1-r_1)x)}{r_1x^2+2x(1-x)+r_2(1-x)^2}    \label{eq:3}
\end{align}

式(\ref{eq:3})を用いて、$y$を$x$の関数としてプロットしたものを図\ref{fig:3}に示す。

\begin{figure}[htbp]
    \centering
    \includegraphics[width=0.6\textwidth]{./data/1/plot2.png}
    \caption{共重合組成曲線}
    \label{fig:3}
\end{figure}

\subsubsection{課題6 Kelen-Tudos法によるモノマー反応性比}

\quad $\alpha, \eta, \xi$を以下のように定義する。

\begin{align*}
    \alpha &= \sqrt{(\frac{F^2}{f})_{max}(\frac{F^2}{f})_{min}} \\
    \eta &= \frac{\frac{F(f-1)}{f}}{a+\frac{F^2}{f}} \\
    \xi &= \frac{\frac{F^2}{f}}{a+\frac{F^2}{f}}
\end{align*}

この定義から、式(\ref{eq:2})は以下のように表せる。

\begin{align}
    \eta = (r_1 + \frac{r_2}{\alpha})\xi - \frac{r_2}{\alpha} \label{eq:4}
\end{align}

式(\ref{eq:4})と、プロットした近似曲線を比較することで反応性比を求めることができる。図\ref{fig:4}に、プロットした近似曲線を示す。

\begin{figure}[htbp]
    \centering
    \includegraphics[width=0.6\textwidth]{./data/1/plot3.png}
    \caption{$\xi$と$\eta$の関係}
    \label{fig:4}
\end{figure}

図\ref{fig:4}より、反応性比は$r_1 = 0.51$, $r_2 = 0.67$となる。

\quad 反応性比が0に近いほど交互共重合体となり、1に近いほどランダム共重合体になる。\cite{1}本実験ではFineman-Ross法でもKelen-Tudos法でも反応性比はおおよそ中間の値をとったことから、得られたポリマーは交互共重合とランダム共重合が入り混じった配列になっていると言える。

\quad 文献値では反応性比は$r_1 = 0.50$, $r_2 = 0.50$であった。\cite{2}。算出した値と比較すると、$r_1$は文献値と近い値をとっているが、$r_2$は文献値よりも大きくなっていることが言える。これは、文献値では反応温度が$60\mathrm{^{\circ}C}$であるのに対し、本実験では$120\mathrm{^{\circ}C}$で反応を進行させた。反応温度が高いと反応が速く進行し、MMAよりもStの方が量的に多く反応することから、$r_2$が大きくなったと考えられる。

\subsubsection{課題7 交互共重合に必要な条件}

\quad 交互共重合を行うためには、単独重合をしないモノマー(反応性比が0に近い)モノマーを用いて反応を進める必要がある。交互共重合体を得やすいモノマーの組み合わせとして、無水マレイン酸とイソブチレンを用いた場合がある。\cite{3}


\subsection{MMAとStのリビングラジカル重合}

\subsubsection{生成物のSECによる測定結果}

\quad 図\ref{fig:5}に、生成物のSECによる測定結果を示す。数平均分子量$M_n=4614$, 分散度$\frac{M_w}{M_n}=1.33$であった。

\begin{figure}[htbp]
    \centering
    \includegraphics[width=0.9\textwidth]{./data/2/sec.png}
    \caption{MMAとStのリビングラジカル重合のSECによる測定結果}
    \label{fig:5}
\end{figure}

\subsubsection{課題1 収率と理論分子量}

\quad 全てポリマーとして導入された場合に得られるポリマーの質量は、$1.60\mathrm{g}$である。また、実際に得られたポリマーの質量は$1.54\mathrm{g}$であった。したがって、収率は以下のようになる。

\begin{align*}
    (収率) = \frac{1.54\mathrm{g}}{1.60\mathrm{g}}\times100\% = 96.3\%
\end{align*}

また、理論分子量は以下のようになる。

\begin{align*}
    (理論分子量) &= (\frac{MMA}{開始剤} \times (MMAの分子量) + \frac{St}{開始剤} \times (Stの分子量)) \times (収率) \\
    &= (70 \times 100\mathrm{g/mol} + 30 \times 104\mathrm{g/mol}) \times 96.3\% \\
    &= 9745\mathrm{g/mol}
\end{align*}

\subsubsection{課題2 SECで得られた分子量との比較}

\quad SECで得られた分子量は4614であるのに対し、理論分子量は9745となり、SECで得られた分子量は理論分子量の半分程度であった。SECでは数平均分子量が測定されるため、重合度が小さい分子が多く含まれていると、数平均分子量が小さくなる。そのため。本実験で得られたポリマーに重合度が小さい分子が多く含まれていたことが考えられる。したがって、SECで得られた分子量は理論分子量の半分程度であったと考えられる。

\quad 実験1のSECで得られた分子量が$M_n=35593$であったのに対し、実験2のSECで得られた分子量は$M_n=4614$であった。実験2の方が実験1よりも分子量が小さいことから、実験2の方が実験1よりも重合度が小さいことが言える。これは、実験1ではラジカル連鎖重合を行なっており、この重合反応ではモノマー反応率によらず数平均分子量が大きくなるためである。一方、実験2ではリビングラジカル重合を行なっており、この重合反応ではモノマー反応率によって数平均分子量が変化する。モノマー反応率が小さいと、数平均分子量は小さくなる。そのため、実験2の方が実験1よりも重合度が小さいと考えられる。

\subsubsection{課題3 リビングラジカル重合の開始と伝播のステップ}

\quad リビングラジカル重合の開始反応を図\ref{fig:6}に示す。開始剤がホモリシス開裂を起こし、ラジカルが生成される。このラジカルがモノマーと反応し、開始反応となる。

\begin{figure}[htbp]
    \centering
    \includegraphics[width=0.9\textwidth]{./data/2/1.png}
    \caption{リビングラジカル重合の開始反応}
    \label{fig:6}
\end{figure}

\quad リビングラジカル重合の成長反応を図\ref{fig:7}に示す。開始反応で生成したドーマント種がホモリシス開裂をおこし、ラジカルが生成される。このラジカルがモノマーと反応し、成長反応となる。

\begin{figure}[htbp]
    \centering
    \includegraphics[width=0.9\textwidth]{./data/2/2.png}
    \caption{リビングラジカル重合の成長反応}
    \label{fig:7}
\end{figure}

\subsection{界面重縮合}

\subsubsection{課題1 ポリマーの収率}

\quad 得られる最大量のポリマーは、ヘキサメチレンジアミン$17\mathrm{mmol}(116\mathrm{g/mol})$に対してセバコイルクロリド$23\mathrm{mmol}(239\mathrm{mol/g})$を用いたことから、$17\mathrm{mmol}$のポリマーと考えられる。重合する際に塩化水素($36.5\mathrm{g/mol}$)が生成することを考慮して、得られるポリマーの最大量は以下のようになる。

\begin{align*}
    (最大量) = (116\mathrm{g/mol} + 239\mathrm{mol/g}) \times 17\mathrm{mmol} - 36.5\mathrm{g/mol} \times 17\mathrm{mmol} \times 2 = 4.79\mathrm{g}
\end{align*}

また、実際に得られたポリマーの質量は$15.8\mathrm{g}$であった。したがって、収率は以下のようになる。

\begin{align*}
    (収率) = \frac{15.8\mathrm{g}}{4.79\mathrm{g}}\times100\% = 330\%
\end{align*}

収率が100\%を超えているのは、生成物に水分が含まれていたためと考えられる。十分に乾燥させることができていなかったため、実際に得られたポリマーの質量が多くなってしまったと考えられる。

\subsubsection{課題2 重合のメカニズム}

\quad 反応機構を図\ref{fig:8}に示す。ヘキサメチレンジアミンがセバコイルクロリドと反応し、ナイロンが生成する。

\begin{figure}[htbp]
    \centering
    \includegraphics[width=0.9\textwidth]{./data/3/1.png}
    \caption{界面重縮合の反応機構}
    \label{fig:8}
\end{figure}

\subsubsection{課題3 ナイロンがなぜ界面に形成されるか}

\quad ヘキサメチレンジアミンは水層、セバコイルクロリドは有機層に溶けている。水層と有機層は分離しているため、界面でのみ反応が進行する。したがって、界面でナイロンが形成される。

\subsubsection{課題4 非化学量論的であるのにもかかわらず重縮合がうまく進み、高分子量ポリマーが得られる理由}

\quad アミド結合は平衡反応である。ポリマーを高分子量で生成させるためにはポリマーを生成する方向に平衡をシフトさせる必要があるが、これは界面で生成したポリマーを引き上げることで平衡が生成する方向にシフトしており、このため非化学量論的であるのにもかかわらず重縮合がうまく進み、高分子量ポリマーが得られると考えられる。


\begin{thebibliography}{99}
    \bibitem{1} 高分子化学 第5版, 共立出版
    \bibitem{2} モノマーの構造と反応性, 大津隆行, \url{https://www.jstage.jst.go.jp/article/yukigoseikyokaishi1943/28/12/28_12_1183/_pdf}
    \bibitem{3} ラジカル重合, 上垣外 正己・佐藤 浩太郎, \url{https://www.jstage.jst.go.jp/article/networkpolymer/30/5/30_234/_pdf/-char/ja}
\end{thebibliography}

\end{document} 