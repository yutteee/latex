\documentclass{ltjsarticle}
\usepackage{amsmath}
\usepackage{amssymb}
\usepackage{graphicx} % Required for inserting images
\usepackage{enumerate}
\usepackage[version=4]{mhchem}
\usepackage{caption}
\usepackage{url}

\title{実験レポテンプレ}
\author{中村 優作}
\date{April 2023}

\pagestyle{plain}
\begin{document}
\flushleft{
    \huge{令和5年度 化学生命工学実験4}
    
    \vspace{100pt}
    
    \huge{層状物質からのナノシートの合成と評価}
}

\vfill
\begin{flushright}

        \Large{\underline{学籍番号 : 082110424}}

        \vspace{30pt}
    
        \Large{\underline{氏名 : 中村優作}}

        \vspace{30pt}
    
        \Large{\underline{実施日 : 2023/12/5, 12/6, 12/7}}
        
\end{flushright}

\newpage

\section{諸言}
\section{実験}

\subsection{酸化物ナノシートの合成及び分析}

\subsubsection{層状酸化物(セラミックス)の合成}

\quad 本実験では、\ce{K_{0.8}Ti_{1.6}Co_{0.4}O4}を目的の層状酸化物としている。\ce{CoO}を$0.461\mathrm{g}$、\ce{TiO2}を$1.96\mathrm{g}$、\ce{K2CO3}を$0.902\mathrm{g}$採取し、乳鉢を使って混合し、30分間粉砕した。粉砕した粉末をアルミナるつぼに入れ、粉砕し終わった粉末を1000度20時間焼成した。

\subsection{グラフェンの合成、分析及び特性評価}

\quad 層状物質をオーブンから取り出し、めのう乳鉢で粉砕した。また、35\%塩酸を用いて$1\mathrm{M}$塩酸を$1\mathrm{L}$調整した。そのと、$250\mathrm{mL}$三角フラスコに$150\mathrm{mL}$の$1\mathrm{M}$\ce{HCl}溶液及び$1.5\mathrm{g}$の層状酸化物を入れてシェイカーで$100\mathrm{rpm}$で1日震盪した。

\section{実験結果}
\section{考察}
\section{課題}
\section{結論}
\section{感想}
\begin{thebibliography}{99}
    % \bibitem{1} hogehoge \url{https://www.google.com/}
\end{thebibliography}

\end{document} 