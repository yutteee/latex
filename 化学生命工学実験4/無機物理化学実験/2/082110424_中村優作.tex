\documentclass{ltjsarticle}
\usepackage{amsmath}
\usepackage{amssymb}
\usepackage{graphicx} % Required for inserting images
\usepackage{enumerate}
\usepackage[version=4]{mhchem}
\usepackage{caption}
\usepackage{url}

\title{実験レポテンプレ}
\author{中村 優作}
\date{April 2023}

\pagestyle{plain}
\begin{document}
\flushleft{
    \huge{令和5年度 化学生命工学実験4}
    
    \vspace{100pt}
    
    \huge{層状物質からのナノシートの合成と評価}
}

\vfill
\begin{flushright}

        \Large{\underline{学籍番号 : 082110424}}

        \vspace{30pt}
    
        \Large{\underline{氏名 : 中村優作}}

        \vspace{30pt}
    
        \Large{\underline{実施日 : 2023/12/5, 12/6, 12/7}}
        
\end{flushright}

\newpage

\section{諸言}

\quad ナノシートは医療、エネルギー、材料など様々な分野への活用が期待されている。ナノ物質を創成する有効なアプローチとして層状化合物の剥離ナノシート化があり、研究が盛んである。本実験では、グラフェンおよび酸化物ナノシートの合成を行い、その特性を評価することを目的とする。

\section{実験}

\subsection{酸化物ナノシートの合成及び分析}

\subsubsection{層状酸化物(セラミックス)の合成}

\quad 本実験では、\ce{K_{0.8}Ti_{1.6}Co_{0.4}O4}を目的の層状酸化物としている。\ce{CoO}を$0.461\mathrm{g}$、\ce{TiO2}を$1.96\mathrm{g}$、\ce{K2CO3}を$0.902\mathrm{g}$採取し、乳鉢を使って混合し、30分間粉砕した。粉砕した粉末をアルミナるつぼに入れ、粉砕し終わった粉末を1000度20時間焼成した。

\subsubsection{層状酸化物の酸処理および剥離}

\quad 層状物質をオーブンから取り出し、めのう乳鉢で粉砕した。また、35\%塩酸$13.35\mathrm{mL}$を用いて$1\mathrm{M}$塩酸を$150\mathrm{mL}$調整した。その後、$250\mathrm{mL}$三角フラスコに$150\mathrm{mL}$の$1\mathrm{M}$\ce{HCl}溶液および$1.5\mathrm{g}$の層状酸化物を入れてシェイカーで$110\mathrm{rpm}$で1日震盪した。回収して上澄を捨てたのちに吸引濾過でサンプルを回収した。得られた層状酸化物に対してXRDを行い評価した。また、電子顕微鏡で層状酸化物にTBAOHを垂らして観察をした。アコーディオン状に層状酸化物が膨潤する様子が観察された。得られた酸処理後の層状酸化物$0.4\mathrm{g}$およびTBAOHを$\ce{TBA+}:\ce{H+}=1:1$になるように$250\mathrm{mL}$三角フラスコに入れた。溶液全体が$100\mathrm{mL}$になる様に定容したのちに、再びシェイカーで1時間浸透することによりナノシートを作成した。
% XRDパターンにより目的とする層状物質が得られたか確認する。

\subsubsection{酸化チタンナノシートの評価に向けたシリコン基盤への転写}

\quad ナノシートをドロップキャスト法により剥離した。サンプルをAFMを用いて評価した。

\subsection{グラフェンの合成、分析及び特性評価}

\quad グラフェンをスコッチテープを用いて一枚一枚剥離していった。テープによる剥離を続けたのちに、得られた破片をシリコン基盤に転写することでナノシートを得た。基盤に転写したナノシートをAFMにより評価をした。

\section{実験結果}

\subsection{層状酸化物の合成}

\quad 焼成前の混合物と、焼成後の層状酸化物の写真を図\ref{fig:before}, \ref{fig:after}に示す。

% 写真を横に2つ並べる
\begin{figure}[htbp]
    \begin{minipage}{0.5\hsize}
        \centering
        \includegraphics[width=5cm]{./result/before.png}
        \caption{焼成前の混合物}
        \label{fig:before}
    \end{minipage}
    \begin{minipage}{0.5\hsize}
        \centering
        \includegraphics[width=5cm]{./result/after.png}
        \caption{焼成後の層状酸化物}
        \label{fig:after}
    \end{minipage}
\end{figure}

\subsection{層状チタン酸化物のXRDパターン}

\quad 層状チタン酸化物の酸処理前と酸処理後の比較したXRDパターンを図\ref{fig:xrd}に示す。

\begin{figure}[htbp]
    \centering
    \includegraphics[width=0.8\linewidth]{./result/xrd.png}
    \caption{層状チタン酸化物のXRDパターン}
    \label{fig:xrd}
\end{figure}

\subsection{グラフェンのAFM}

\quad グラフェンのAFMを図\ref{fig:afm}に示す。

\begin{figure}[htbp]
    \centering
    \includegraphics[width=0.8\linewidth]{./result/afm.png}
    \caption{グラフェンのAFM}
    \label{fig:afm}
\end{figure}

\subsection{層状チタン酸化物のAFM}

\quad 層状チタン酸化物のAFMを図\ref{fig:afm2}に示す。

\begin{figure}[htbp]
    \centering
    \includegraphics[width=0.8\linewidth]{./result/afm2.png}
    \caption{層状チタン酸化物のAFM}
    \label{fig:afm2}
\end{figure}

\section{考察}

\subsection{層状酸化物とTBAOHの反応}

\quad 層状酸化物とTBAOHの反応によって、層状酸化物の層間距離が広がったと考えられる。層状酸化物は、層間に\ce{H3O+}が存在することによって層間距離が狭まっている。層状酸化物とTBAOHの反応によって、\ce{K+}が\ce{TBA+}に置換されることによって層間距離が広がったと考えられる。しかし、\ce{TBA+}の大きさは 10\AA 程度であり、アコーディオン状に広がった時の様子から、\ce{TBA+}の大きさ以上に層間距離が広がっていたと考えられる。これは、\ce{TBA+}が層間に入ると同時に大量の水が取り込まれ、\ce{TBA+}との水素結合による層間距離の広がりや、層間と溶媒の間での浸透圧の差による水の入り込みが要因と考えられる。\cite{1}
% https://www.jst.go.jp/pr/announce/20130328/index.html

\subsection{層状酸化物の剥離}

\quad 課題5.2でも述べるが、得られたシートは単層ではなく、3層から4層程度のものであり、十分に剥離が行えていないことが言える。本実験ではシェイカーで1時間浸透したが、この浸透が十分でなかったと考えられるため、浸透時間を長くすることでより剥離が行えると考えられる。また、層状酸化物で\ce{K+}が置き換えきれていないことも考えられるが、XRDパターンを見ると、層状酸化物のピークが消えていることから、\ce{K+}は十分に置き換わっていると言える。正確に組成を評価するには、ICP発光分析法などを用いることでより詳細な組成評価が可能である。

\subsection{グラフェンの剥離}

\quad グラフェンの剥離は複数回行ったが、単層を得るのは困難であった。一度はがすだけでは厚みの大きな層が得られるため何度もはがす必要があった。しかし何度もはがすことによって面積の小さい層になってしまい、ナノシートが得られたかを確認するのが困難であった。基盤に転写する際にも、層同士のファンデルワールス力よりも基盤と層の相互作用の方を強くするために、強く擦り付ける必要があったりと、機械的剥離での限界があると思われた。

\quad グラフェンの化学的剥離をする方法もある。グラファイトを酸化剤と反応させることで、溶媒中に酸化グラフェンを分散させることができる。機械的剥離では大量合成が困難だと考えられたが、化学的剥離を行うことで大量合成が可能となる。しかし、グラフェンに戻すには還元を行う必要があるが、還元をしても完全なグラフェンには戻らないという問題がある。\cite{2}

% https://www.shingi.jst.go.jp/pdf/2015/6-univ04.pdf

\section{課題}

\subsection{酸化物層状物質の酸処理の過程で何が進⾏しているのか、実際に測定したサンプルのXRD パターンをもとに考察せよ。}

\quad XRDパターンから、積層構造のピークの$2\theta$の値が小さくなっていることが読み取れる。ブラッグの式から、$\theta$が小さくなると層間距離dが大きくなるため、酸処理前後で層間距離が広がったと言える。これは、層間に存在する\ce{K+}がよりサイズの大きい\ce{H3O+}に置換されたことによるものと考えられる。

\subsection{今回 AFM で測定した厚みから、今回得られたナノシートが何層重なっているか類推せよ。また、酸化物ナノシートとグラフェンにおいて、異なる剥離⽅法を⽤いた理由について考察し、どのような⼿法が層状物質の剥離に適応可能かそこから類推せよ。}

\quad グラフェンのAFMの結果より、厚みは$2\mathrm{nm}$ほどであった。炭素原子の大きさは1原子あたり 4\AA 程度であり、基盤と層状物質の間に水などが含まれることを考えると、層状物質は3層から4層程度のものだと考えられる。

\quad また、層状チタン酸化物のAFMの結果より、厚みは$3\mathrm{nm}$ほどであった。1層の厚みは$0.7\mathrm{nm}$程度で層間が$0.3\mathrm{nm}$程度であることを考えると、層状チタン酸化物は2層から3層程度のものだと考えられる。

\quad グラフェンと層状酸化物では、層間に働く結合力の種類が異なる。グラフェンは層同士のファンデルワールス力により結合をしているのに対し、層状酸化物ではイオン結合により結合をしている。このため、グラフェンは層間の力が比較的弱く、スコッチテープによる引き剥がしの様な物理的な力で剥離をすることが可能である。一方で層状酸化物では、イオン結合により層間の結合が強く、物理的な力での剥離は困難である。今回は、層状酸化物の層間を広げることで、層間の結合を弱めることにより剥離を行った。
% TODO: 他の層状物質の剥離方法について調べる。
\subsection{ナノシートの結晶構造や組成をより詳細に評価したい場合にはどのような分析⽅法が考えられるのか理由とともに述べよ。}

\quad 組成をより詳細に評価するための分析法として、ICP発光分析法がある。ICPとは誘導結合プラズマの略であり、一般的にはアルゴンプラズマを指す。アルゴンプラズマは極めて高温であり、炎を用いる分析法では励起されない一部の元素も励起して発光することがかのうなため、ほとんど全ての元素の発光分析が可能である。本実験で持ちた酸化チタンナノシートの組成を評価するためには、ICP発光分析法を用いることで、より詳細な組成評価が可能である。\cite{3}
% 5章

\quad また、電子顕微鏡を用いた微細構造の観察なども有効と考えられる。透過型電子顕微鏡(TEM)は観察領域に均一な電子線を照射し、物体を通過した電子線をレンズで投影して内部構造を観察する装置である。電子線回折像から結晶構造と格子定数を求めることができる。また、透過波と回折波の干渉でも像コントラストが形成され、結晶の原子配列に相当する像を得ることができる。\cite{4}
% 10章

\subsection{得られたナノシートはどのような材料に応⽤することができると考えられるか、理由と共に述べよ。}

\quad 酸化チタンは光触媒としての作用を持ち、光により励起された電子が他の分子に結合して還元を起こし、酸化チタン自体は酸化される。特に酸化チタンは超親水性という特徴を持っており、酸化チタンの酸素が光照射によって抜け落ちて水分子と反応して水酸基を作ることにより表面と水分の馴染みが良くなる。これによって汚れが洗い流されやすくなる。このような性質を持つ酸化チタンは、自動車の窓ガラスや外壁材、衛生陶器などに応用されている。\cite{}
% https://www.u-tokyo.ac.jp/focus/ja/features/f_00057.html

\section{結論}

\quad 本実験では純粋な単層のナノシートを得ることができなかった。化学的剥離を行うこと、浸透の時間を長くすることなどによって、より純粋な単層のナノシートを得ることができると考えられる。また、本実験で得られたナノ物質の組成を理解するために、ICP発光分析法や、電子顕微鏡を用いた観察などを行うことが有効であると考えられる。

\section{感想}

\quad 特にTBAOHを用いた層状酸化物の剥離において、層状酸化物の層間距離が広がる様子を観察することができて面白かった。単層を大量に得ることの難しさも感じたが、そこをどう工夫するかについての興味が湧いた。

\begin{thebibliography}{99}
    \bibitem{1} 水による層状結晶のきわめて珍しい巨大膨潤現象を発見, 独立行政法人 物質・材料研究機構, \url{https://www.jst.go.jp/pr/announce/20130328/index.html}
    \bibitem{2} 酸化グラフェンの実用化を指向した合成法と用途開拓, 仁科勇太, 岡山大学異分野融合先端研究コア, \url{https://www.shingi.jst.go.jp/pdf/2015/6-univ04.pdf}
    \bibitem{3} 機器分析, 大谷 肇, 講談社, 5章
    \bibitem{4} 機器分析, 大谷 肇, 講談社, 10章
    \bibitem{5} 光触媒の新世界, 東京大学, \url{https://www.u-tokyo.ac.jp/focus/ja/features/f_00057.html}
\end{thebibliography}

\end{document} 